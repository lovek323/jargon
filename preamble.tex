\texttt{\#======= THIS IS THE JARGON FILE, VERSION 4.2.2, 20 AUG 2000 =======\#}

\texttt{\ 010}

\texttt{\ 001}

\texttt{\ 111}
\setlength{\parskip}{6pt}

This is the Jargon File, a comprehensive compendium of hacker slang illuminating many aspects of hackish tradition, folklore, and humor.

This document (the Jargon File) is in the public domain, to be freely used, shared, and modified. There are (by intention) no legal
restraints on what you can do with it, but there are traditions about its proper use to which many hackers are quite strongly attached.
Please extend the courtesy of proper citation when you quote the File, ideally with a version number, as it will change and grow over time.
(Examples of appropriate citation form: ``Jargon File 5.0.1'' or ``The on-line hacker Jargon File, version 5.0.1, 5 JAN 2012''.)

The Jaron File is a common heritage of the hacker culture. Over the years a number of individuals have volunteered considerable time to
maintaining the File and been recognized by the net at large as editors of it. Editorial responsibilities include: to collate contributions
and suggestions from others; to seek out corroborating information; to cross-reference related entries; to keep the file in a consistent
format; and to announce and distribute updated versions periodically. Current volunteer editors include:

Yash Tulsyan \url{yashtulsyan@gmail.com}

Although there is no requirement that you do so, it is considered good form to check with an editor before quoting the File in a published
work or commercial product. We may have additional information that would be helpful to you and can assist you in framing your quote to
reflect not only the letter of the File but its spirit as well.

All contributions and suggestions about this file sent to a volunteer editor are gratefully received and will be regarded, unless otherwise
labelled, as freely given donations for possible use as part of this public-domain file.

From time to time a snapshot of this file has been polished, edited, and formatted for commercial publication with the cooperation of the
volunteer editors and the hacker community at large. If you wish to have a bound paper copy of this file, you may find it convenient to
purchase one of these. They often contain additional material not found in on-line versions. The two `authoriezd' editions so far are
described in the Revision History section; there may be more in the future.

