From the early 1980s onward, a flourishing culture of local, MS-DOS-based bulletin boards has been developing separately from Internet
hackerdom. The BBS culture has, as its seamy underside, a stratum of `pirate boards' inhabited by \citeentry{cracker}s, phone phreaks, and
\citeentry{warez d00dz}. These people (mostly teenagers running IBM-PC clones from their bedrooms) have developed their own characteristic
jargon, heavily influenced by skateboard lingo and underground-rock slang.

Though crackers often call themselves `hackers', they aren't (they typically have neither significant programming ability, nor Internet
expertise, nor experience with UNIX or other true multi-user systems). Their vocabulary has little overlap with hackerdom's. Nevertheless,
this lexicon covers much of it so the reader will be able to understand what goes by on bulletin-board systems.

Here is a brief guide to cracker and \citeentry{warez d00dz} usage:

\begin{itemize}
	\item Misspell frequently. The substitutions

		\begin{quote}
			phone $\Rightarrow$ fone\\
			freak $\Rightarrow$ phreak
		\end{quote}

		are obligatory.
	\item Always substitute `z's for `s's. (i.e. ``codes'' $\rightarrow$ ``codez'').
	\item Type random emphasis charactrs after a post line (i.e. ``Hey Dudes!\#!$\#$!\#!\$'').
	\item Use the emphatic `k' prefix (``k-kool'', ``k-rad'', ``k-awesome'') frequently.
	\item Abbreviate compulsively (``I got lotsa warez w/ docs'').
	\item Substitute `0' for `o' (``r0dent'', ``l0zer'').
	\item TYPE ALL IN CAPS LOCK, SO IT LOOKS LIKE YOU'RE YELLING ALL THE TIME.
\end{itemize}

These traits are similar to those of \citeentry{B1FF}, who originated as a parody of naive \citeentry{BBS} users. Occasionally, this sort
of distortion may be used as heavy sarcasm by a real hacker, as in:

\begin{verbatim}
	> I got X windows running under Linux!

	d00d!  u R an \1337 hax0r
\end{verbatim}

The only practice resembling this in actual hacker usage is the substitution of a dollar sign of `s' in names of products or service felt
to be excessively expensive, e.g., Compu\$serve, Micro\$oft.

For further discussion of the pirate-board subculture, see \citeentry{lamer}, \citeentry{elite}, \citeentry{leech}, \citeentry{poser},
\citeentry{cracker}, and especially \citeentry{warez d00dz}. Although cracker slang (which they call leetspeak, out of a corruption of the
word ``elite'') is distinct from hacker jargon, some leetspeak terms are piped from hacker fargon and then transformed according to the
above rules. For example, newbie became ``n00b''. Recently, leetspeak has become mainstream and mutated into a further form called
LOLSPEAK, which is generally derived from shorthand used in text messaging (SMS). Again, these are distinct from hacker slang, though some
have produced an unholy hybrid known as LOLCODE (as well as lolbash), an esoteric programming language which uses LOLSPEAK in its syntax.

