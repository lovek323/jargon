This document is a collection of slang terms used by various subcultures of computer hackers. Though some technical material is
included for background and flavor, it is not a technical dictionary; what we describe here is the language hackers use among
themselves for fun, social communication, and technical debate.

The `hacker culture' is actually a loosely networked collection of subcultures that is nevertheless conscious of some important shared
experiences, shared roots, and shared values. It has its own myths, heroes, villains, folk epics, in-jokes, taboos, and dreams. Because
hackers as a group are particularly creative people who define themselves partly by rejection of `normal' values and working habits, it
has unusually rich and conscious traditions for an intentional culture less than 60 years old.

As usual with slang, the special vocabulary of hackers helps hold their culture together -- it helps hackers recognize each other's
places in the community and expresses shared values and experiences. Also as usual, \textit{not} knowing the slang (or using it
inappropriately) defines one as an outsider, a mundane, or (worst of all in hackish vocabulary) possibly even a \citeentry{suit}. All
human cultures use slang in this threefold way -- as a tool of communication, and of onclusion, and of exclusion.

Among hackers, though, slang has a subtler aspect, paralleled perhaps in the slang of jazz musicians and some kinds of fine artists but
hard to detect in most technical or scientific cultures; parts of it are code for shared states of \textit{consciousness}. There is a
whole range of altered states and problem-solving mental stances basic to high-level hacking which don't fit into conventional
linguistic reality any better than a Coltrane solo or one of Maurits Escher's `trompe l'oeil' compositions (Escher is a favorite of
hackers), and hacker slang encodes these subtleties in many unobvious ways. As a simple example, take the distinction between a
\citeentry{kludge} and an \citeentry{elegant} solution, and the differing connotations attached to each. The distinction is not only of
engineering significance; it reaches right back into the nature of the generative process in program design and asserts something
important about two different kinds of relationship between the hacker and the hack. Hacker slang is unusually rich in implications of
this kind, of overtones and undertones that illuminate the hackish psyche.

But there is more. Hackers, as a rule, love wordplay and are very conscious and inventive in their use of language. These traits seem
to be common in young children, but the conformity-enforcing machine we are pleased to call an educational system bludgeons them out of
most of us before adolescence. Thus, linguistic invention in most subcultures of the modern West is a halting and largely unconscious
process. Hackers, by contrast, regard slang formation and use as a game to be played for conscious pleasure. Their inventions thus
display an almost unique combination of the neotenous enjoyment of language-play with the discrimination of educated and powerful
intelligence. Further, the electronic media which knit them together are fluid, `hot' connections, well adapted to both the
dissemination of new slang and the ruthless culling of weak and superannuated specimens. The results of this process give us perhaps a
uniquely intense and accelerated view of linguistic evolution in action.

Hacker slang also challenges some common linguistic and anthropological assumptions. For example, it has recently become fashionable to
speak of `low-context' versus `high-context' communication, and to classify cultures by the preferred context level of their languages
and art forms. It is usually claimed that low-context communication (characterized by precision, clarity, and completeness of
self-contained utterances) is typical in cultures which value logic, objectivity, individualism, and competition; by contrast,
high-context communication (elliptical, emotive, nuance-filled, multi-modal, heavily coded) is associated with cultures which value
subjectivity, consensus, cooperation, and tradition. What then are we to make of hackerdom, which is themed around extremely
low-context interaction with computers and exhibits primarily ``low-context'' values, but cultivates an almost absurdly high-context
slang style?

The intensity and consciousness of hackish invention make a compilation of hacker slang a particularly effective window into the
surrounding culture -- and, in fact, this one is the latest version of an evolving compilation called the `Jargon File', maintained by
hackers themselves for over 36 years. This one (like its ancestors) is primarily a lexicon, but also includes topic entries which
collect background or sidelight information on hacker culture that would be awkward to try to subsume under individual slang
definitions.

Though the format is that of a reference volume, it is intended that the material be enjoyable to browse. Even a complete outsider
should find at least a chuckle on nearly every page, and much that is amusingly though-provoking. But it is also true that hackers use
humorous wordplay to make strong, sometimes combative statements about what they feel. Some of these entries reflect the views of
opposing sides in disputes that have been genuinely passionate; this is deliberate. We have not tried to moderate or pretty up these
disputes; rather we have attempted to ensure that \textit{everyone's} sacred cows get gored, impartially. Compromise is not
particularly a hackish virtue, but the honest presentation of divergent viewpoints is.

The reader with minimal computer background who finds some references incomprehensibly technical can safely ignore them. We have not
felt it either necessary or desirable to eliminate all such; they, too, contribute flavor, and one of this document's major intended
audiences -- fledgling hackers already partway inside the culture -- will benefit from them.

A selection of longer items of hacker folklore and humor is included in \citeappendix{Appendix A}. The `outside' reader's attention is
particularly directed to the Portrait of J. Random Hacker in \citeappendix{Appendix B}. Appendix C, the \citeappendix{Bibliography}, lists
some non-technical works which have either influenced or described the hacker culture.

Because hackerdom is an intentional culture (one each individual must chose by action to join), one should not be surprised that the
line between description and influence can become more than a little blurred. Earlier versions of the Jargon File have played a central
role in spreading hacker language and the culture that goes with it to successively larger populations, and we hope and expect that
this one will do likewise.

