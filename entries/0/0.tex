\mainentry{0}

Numeric zero, as opposed to the letter `O' (the 15th letter of the
English alphabet). In their unmodified forms they look a lot alike, and
various kluges invented to make them visually distinct have compounded
the confusion. If your zero is center-dotted and letter-O is not, or if
letter-O looks almost rectangular but zero looks more like an American
football stood on end (or the reverse), you're probably looking at an
old-style ASCII graphic set descended from the default typewheel on the
venerable ASR-33 Teletype (Scandinavians, for whom \O is a letter, curse
this arrangement). (Interestingly, the slashed zero long predates
computers; Florian Cajori's monumental ``A History of Mathematical
Notations'' notes that it was used in the twelfth and thirteenth
centuries.) If letter-O has a slash across it and the zero does not,
your display is tuned for a very old convention used at IBM and a few
other early mainframe makers (Scandanavians curse this arrangement even
more, because it means two of their letters collide). Some
Burroughs/Unisys equipment displays a zero with a reversed slash. Old
CDC computers rendered letter-O\opt{changes}{\footnote{This was `letter
O' without the hyphen in the source file.}} as an unbroken oval and 0 as
an oval broken at upper right and lower left. And yet another convention
common on early line printers left zero unornamented but added a tail or
hook to the letter-O so that it resembled an inverted Q or cursive
capital letter-O (this was endorsed by a draft ANSI standard for how to
draw ASCII characters, but the final standard changed the distinguisher
to a tick-mark in the upper-left corner). Are we sufficiently confused
yet?

