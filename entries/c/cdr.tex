\mainentry{cdr} /ku'dr/ or /kuh'dr/ vt.

[from LISP] To skip past the first item from a list of things (generalized from
the LISP operation on binary tree structures, which returns a list consisting of
all but the first element of its argument). In the form `cdr down', to trace
down a list of elements: \inlinequote{Shall we cdr down the agenda?} Usage:
silly. See also \citeentry{loop through}.

Historical note: The instruction format of the IBM 704 that hosted the original
LISP implementation featured two 15-bit fields called the `address' and
`decrement' parts. The term `cdr' was originally `Contents of Decrement part of
Register'. Similarly, `car' stood for `Contents of Address part of Register'.

The cdr and car operations have since become bases for formation of compound
metaphors in non-LISP contexts. GLS recalls, for example, a programming project
in which strings were represented as linked lists; the get-character and
skip-character operations were of course called CHAR and CHDR.

