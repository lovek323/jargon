\mainentry{chad} /chad/ n.

\begin{inparaenum}
\item {[}common] The perforated edge strips on printer paper, after they have
    been separated from the printed portion. Also called \citeentry{salvage},
    \citeentry{perf}, and \citeentry{ripoff}.
\item {[}obs.]\Changes{Was not in brackets, which was inconsistent.} The
    confetti-like paper bits punched out of cards or paper tape; this has also
    been called `chaff', `computer confetti', and `keypunch droppings'. It's
    reported that this was very old Army slang, and it may now be mainstream; it
    has been reported seen (1993) in directions for a card-based voting machine
    in California.
\end{inparaenum}

Historical note: One correspondent believes `chad' (sense 2) derives from the
Chadless keypunch (named for its inventor), which cut little u-shaped tabs in
the card to make a hole when the tab folded back, rather than punching out a
circle/rectangle; it was clear that if the Chadless keypunch didn't make them,
then the stuff that other keypunches made had to be `chad'. There is a legend
that the word was originally acronymic, standing for ``Card Hole Aggregate
Debris'', but this has all the earmarks of a \citeentry{backronym}.

