\mainentry{case and paste} n.

[from `cut and paste'] The addition of a new \citeentry{feature} to an existing
system by selecting the code from an existing feature and pasting it in with
minor changes. Common in telephony circles because most operations in a
telephone switch are selected using case statements. Leads to
\citeentry{software bloat}.\Changes{Was in an in-paragraph list with only one
item.}

In some circles of EMACS users this is called `programming by Meta-W', because
Meta-W is the EMACS command for copying a block of text to a kill buffer in
preparation for\Changes{Was `to', but that didn't make sense.} pasting it in
elsewhere. The term is condescending, implying that the programmer is acting
mindlessly rather than thinking carefully about what is required to integrate
the code for two similar cases.

At \citeentry{DEC} (now Compaq), this is sometimes called `clone-and-hack'
coding.

