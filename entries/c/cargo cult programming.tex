\mainentry{cargo cult programming} n.

A style of (incompetent) programming dominated by ritual inclusion of code or
program structures that serve no real purpose. A cargo cult programmer will
usually explain the extra code as a way of working around some bug encountered
in the past, but usually neither the bug nor the reason the code apparently
avoided the bug was ever fully understood (compare \citeentry{shotgun
debugging}, \citeentry{voodoo programming}).

The term `cargo cult' is a reference to aboriginal relations that grew up in the
South Pacific after World War II. The practices of these cults center on
building elaborate mockups of airplanes and military style landing strips in the
hope of bringing the return of the god-like airplanes that brought such
marvelous cargo during the war. Hackish usage derives from Richard Feynman's
characterization of certain practices as ``cargo cult science'' in his book
``Surely You're Joking, Myr. Feynman!'' (W. W. Norton \& Co., New York 1985,
ISBN 0-393-01921-7).

