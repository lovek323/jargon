\mainentry{C} n.

\begin{inparaenum}
\item The third letter of the English alphabet.
\item ASCII 1000011.
\item The name of a programming language designed by Dennis Ritchie during the
    early 1970s and immediately used to reimplement \citeentry{Unix}; so called
    because many features derived from an earlier compiler named `B' in
    commemoration of its parent, BCPL. (BCPL was in turn descended from an
    earlier Algol-derived language, CPL.) Before Bjarne Stroustrup settled the
    question by designing \citeentry{C++}, there was a humorous debate over
    whether C's successor should be named `D' or `P'. C became immensely popular
    outside Bell Labs after about 1980 and is now the dominant language in
    systems and microcomputer applications programming. See also
    \citeentry{languages of choice}, \citeentry{indent style}.
\end{inparaenum}

C is often described, with a mixture of fondness and disdain varying according
to the speaker, as ``a language that combines all the elegance and power of
assembly language with all the readability and maintainability of assembly
language''.

