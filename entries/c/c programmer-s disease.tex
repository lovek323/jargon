\mainentry{C Programmer's Disease} n.

The tendency of the undisciplined C programmer to set arbitrary but supposedly
generous static limits on table sizes (defined, if you're lucky, by constants in
header files) rather than taking the trouble to do proper dynamic storage
allocation. If an application user later needs to put 68 elements into a table
of size 50, the afflicted programmer reasons that he or she can easily reset the
table size to 68 (or even as much as 70, to allow for future expansion) and
recompile. This gives the programmer the comfortable feeling of having made the
effort to satisfy the user's (unreasonable) demands, and often affords the user
multiple opportunities to explore the marvelous consequences of
\citeentry{fandango on core}. In severe cases of the disease, the programmer
cannot comprehend why each fix of this kind seems only to further disgruntle the
user.

