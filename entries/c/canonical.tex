\mainentry{canonical} adj.

[very common; historically, `according to religious law'] The usual or standard
state or manner of something. This word has a somewhat more technical meaning in
mathematics. Two formulas such as $9+x$ and $x+9$ are said to be equivalent
because they mean the same thing, but the second one is in `canonical form'
because it is written in the usual way, with the highest power of $x$ first.
Usually there are fixed rules you can use to decide whether something is in
canonical form. The jargon meaning, a relaxation of the technical meaning,
acquired its present loading in computer-science culture largely through its
prominence in Alonzo Church's work in computation theory and mathematical logic
(see \citeentry{Knights of the Lambda Calculus}). Compare \citeentry{vanilla}.

Non-technical academics do not use the adjective `canonical' in any of the
senses defined above with any regularity; they do however use the nouns `canon'
and `canonicity' (not canonicalness or canonicality\Changes{Was originally `not
**canonicalness or **canonicality', but I couldn't work out why the asterisks
were put there.}). The `canon' of a given author is the complete body of
authentic works by that author (this usage is familiar to Sherlock Holmes fans
as well as to literary scholars). `The canon' is the body of works in a given
field (e.g., works of literature, or of art, or of music) deemed worthwhile for
students to study and for scholars to investigate.

The word `canon' has an interesting history. It derives ultimately from the
Greek `kanon' (akin to the English `cane') referring to a reed. Reeds were used
for measurement, and in Latin and later Greek the word `canon' meant a rule or a
standard. The establishment of a canon of scriptures within Christianity was
meant to define a standard or a rule for the religion. The above non-techspeak
academic usages stem from this instance of a defined and accepted body of work.
Alongside this usage was the promulgation of `canons' (`rules') for the
government of the Catholic Church. The techspeak usages (``according to
religious law'') derive from this use of the Latin `canon'.

Hackers invest this term with a playfulness that makes an ironic contrast with
its historical meaning. A true story: One Bob Sjoberg, new at the MIT AI Lab,
expressed some annoyance at the incessant use of jargon. Over his loud
objections, GLS and RMS made a point of using as much of it as possible in his
presence, and eventually it began to sink in. Finally, in one conversation, he
used the word `canonical' in jargon-like fashion without thinking. Steele:
``Aha! We've finally got you talking jargon too!'' Stallman: ``What did he
say?'' Steele: ``Bob just used `canonical' in the canonical way.''

Of course, canonicality depends on context, but it is implicitly defined as the
way hackers normally expect things to be. Thus, a hacker may claim with a
straight face that `according to religious law' is not the canonical meaning of
`canonical'.

