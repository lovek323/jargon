\mainentry{cat} [from `catenate' via \citeentry{Unix} cat(1)] vt.

\begin{inparaenum}
\item [techspeak] To spew an entire file to the screen or some other output sink
    without pause.
\item By extension, to dump large amounts of data at an unprepared target or
    with no intention of browsing it carefully. Usage: considered silly. Rare
    outside Unix sites\Changes{The editor has noticed this being used somewhat
    frequently outside Unix sites and so it is unclear whether this distinction
    still applies.}. See also \citeentry{dd}, \citeentry{BLT}. 
\end{inparaenum}

Among Unix fans, cat(1) is considered an excellent example of user-interface
design, because it delivers the file contents without such verbosity as spacing
or headers between the files, and because it does not require the files to
consist of lines of text, but works with any sort of data.

Among Unix haters, cat(1) is considered the \citeentry{canonical} example of bad
user-interface design, because of its woefully unobvious name. It is far more
often used to \citeentry{blast} a file to standard output than to concatenate
two files. The name for cat for the former operation is just as unintuitive as,
say, LISP's \citeentry{cdr}.

Of such oppositions are \citeentry{holy wars} made\dots

