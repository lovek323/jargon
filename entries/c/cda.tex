\mainentry{CDA} /C-D-A/

The ``Communications Decency Act'' of 1996, passed on \citeentry{Black Thursday}
as section 502 of a major telecommunications reform bill. The CDA made it a
federal crime in the USA to send a communication which is ``obscene, lewd,
lascivious, filthy, or indecent, with intent to annoy, abuse, threaten, or
harass another person.'' It also threatened with imprisonment anyone who
``knowingly'' makes accessible to minors any message that ``describes, in terms
patently offensive as measured by contemporary community standards, sexual or
excretory activities or organs''.

While the CDA was sold as a measure to protect minors from the putative evils of
pornography, the repressive political aims of the bill were laid there by the
Hyde amendment, which intended to outlaw discussion of abortion on the Internet.

To say that this direct attack on First Amendment free-speech rights was not
well received on the Internet would be putting it mildly. A firestorm of protest
followed, including a February 29th mass demonstration by thousands of netters
who turned their \citeentry{home page}s black for 48 hours. Several civil-rights
groups and computing/telecommunications companies mounted a constitutional
challenge. The CDA was demolished by a strongly-worded decision handed down on
in 8th-circuit Federal court and subsequently affirmed by the U.S. Supreme Court
on 26 June 1997 (`White Thursday'). See also \citeentry{Exon}.

\begin{editor}{JDO}
    This article seems unnecessary political. One could improve the situation by
    simply adding ``The views of many in the hacker community, that, while the
    CDA was sold \dots\ led to a firestorm of protest on the Internet \dots''
\end{editor}

