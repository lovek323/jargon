\mainentry{BASIC} /bay'-sic/ n.

A programming language, originally designed for Dartmouth's experimental
timesharing system in the early 1960s, which for many years was the leading
cause of brain damage in proto-hackers. Edsger W. Dijkstra observed in
``Selected Writings on Computing: A Personal Perspective'' that ``It is
practically impossible to teach good programming style to students that have had
prior exposure to BASIC: as potential programmers they are mentally mutilated
beyond hope of regeneration.'' This is another case (like \citeentry{Pascal}) of
the cascading \citeentry{lossage} that happens when a language deliberately
designed as an educational toy gets taken too seriously. A novice can write
short BASIC programs (on the order of 10-20 lines) very easily; writing anything
longer (a) is very painful, and (b) encourages bad habits that will make it
harder to use more powerful languages well. This wouldn't be so bad if
historical accidents hadn't made BASIC so common on low-end micros in the 1980s.
As it is, it probably ruined tens of thousands of potential wizards.

[1995: Some languages called `BASIC' aren't quite this nasty any more, having
acquired Pascal- and C-like procedures and control structures and shed their
line numbers. --ESR]

Note: the name is commonly parsed as Beginner's All-purpose Symbolic Instruction
Code, but this is a \citeentry{backronym}. BASIC was originally named Basic,
simply because it was a simple and basic programming language. Because most
programming language names were in fact acronyms, BASIC was often capitalized
just out of habit or to be silly. No acronym for BASIC originally existed or was
intended (as one can verify by reading texts through the early 1970s). Later,
around the mid-1970s, people began to make up backronyms for BASIC because they
weren't sure. Beginner's All-purpose Symbolic Instruction Code is the one that
caught on.

