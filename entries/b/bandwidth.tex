\mainentry{bandwidth} n.

\begin{inparaenum}
    \item {[}common] Used by hackers (in a generalization of its technical
        meaning) as the volume of information per unit time that a computer,
        person, or transmission medium can handle. ``Those are amazing graphics,
        but I missed some of the detail -- not enough bandwidth, I guess.''
        Compare \citeentry{low-bandwidth}. This generalized usage began to go
        mainstream after the Internet population explosion of 1993--1994.
    \item Attention span.
    \item On \citeentry{Usenet}, a measure of network capacity that is often
        wasted by people complaining about how items posted by others are a
        waste of bandwidth.
\end{inparaenum}

