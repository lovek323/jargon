\mainentry{barf} /barf/ n.,v.

[common; from mainstream slang meaning `vomit']
\begin{inparaenum}
    \item interj. Term of disgust. This is the closest hackish equivalent of the
        Valspeak ``gag me with a spook''. (Like, euwww!) See \citeentry{bletch}.
    \item vi. To say ``Barf!'' or emit some similar expression of disgust. ``I
        showed him my latest hack and he barfed'' means only that he complained
        about it, not that he literally vomited.
    \item vi. To fail to work because of unacceptable input, perhaps with a
        suitable error message, perhaps not. Examples: ``The division operation
        barfs if you try to divide by 0.'' (That is, the division operation
        checks for an attempt to divide by zero, and if one is encountered it
        causes the operation to fail in some unspecified, but generally obvious,
        manner.) ``The text editor barfs if you try to read in a new file before
        writing out the old one.'' See \citeentry{choke}, \citeentry{gag}.
\end{inparaenum}

 In Commonwealth Hackish, `barf' is generally replaced by `puke' or `vom'.
\citeentry{Barf} is sometimes also used as a \citeentry{metasyntactic variable},
like \citeentry{foo} or \citeentry{bar}.  

