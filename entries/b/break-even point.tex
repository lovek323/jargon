\mainentry{break-even point} n.

In the process of implementing a new computer language, the point at which the
language is sufficiently effective that one can implement the language in
itself. That is, for a new language called, hypothetically, FOOGOL, one has
reached break-even when one can write a demonstration compiler for FOOGOL in
FOOGOL, discard the original implementation language, and thereafter use working
versions of FOOGOL to develop newer ones. This is an important milestone; see
\citeentry{MFTL}.

Since this entry was first written, several correspondents have reported that
there actually was a compiler for a tiny Algol-like language called Foogol
floating around on various \citeentry{VAXen} in the early and mid-1980s. A
FOOGOL implementation is available at the Retrocomputing Museum
\url{http://www.ccil.org/retro}.

