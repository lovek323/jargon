\mainentry{bug} n.

An unwanted and unintended property of a program or piece of hardware, esp. one
that causes it to malfunction. Antonym of \citeentry{feature}. Examples:
``There's a bug in the editor: it writes things out backwards.'' ``The system
crashed because of a hardware bug.'' ``Fred is a winner, but he has a few bugs''
(i.e., Fred is a good guy, but he has a few personality problems).

Historical note: Admiral Grace Hopper (an early computing pioneer better known
for inventing \citeentry{COBOL}) liked to tell a story in which a technician
solved a \citeentry{glitch} in the Harvard Mark II machine by pulling an actual
insect out from between the contacts of one of its relays, and she subsequently
promulgated \citeentry{bug} in its hackish sense as a joke about the incident
(though, as she was careful to admit, she was not there when it happened). For
many years the logbook associated with the incident and the actual bug in
question (a moth) sat in a display case at the Naval Surface Warfare Center
(NSWC). The entire story, with a picture of the logbook and the moth taped into
it, is recorded in the ``Annals of the History of Computing'', Vol. 3, No. 3
(July 1981), pp. 285-286.

The text of the log entry (from September 9, 1947), reads ``1545 Relay \#70
Panel F (moth) in relay. First actual case of bug being found''. This wording
establishes that the term was already in use at the time in its current specific
sense -- and Hopper herself reports that the term `bug' was regularly applied to
problems in radar electronics during WWII.

Indeed, the use of `bug' to mean an industrial defect was already established in
Thomas Edison's time, and a more specific and rather modern use can be found in
an electrical handbook from 1896 (``Hawkin's New Catechism of Electricity'',
Theo. Audel \& Co.) which says: ``The term `bug' is used to a limited extent to
designate any fault or trouble in the connections or working of electric
apparatus.'' It further notes that the term is ``said to have originated in
quadruplex telegraphy and have been transferred to all electric apparatus.''

The latter observation may explain a common folk etymology of the term; that it
came from telephone company usage, in which ``bugs in a telephone cable'' were
blamed for noisy lines. Though this derivation seems to be mistaken, it may well
be a distorted memory of a joke first current among telegraph operators more
than a century ago!

Or perhaps not a joke. Historians of the field inform us that the term ``bug''
was regularly used in the early days of telegraphy to refer to a variety of
semi-automatic telegraphy keyers that would send a string of dots if you held
them down. In fact, the Vigroplex keyers (which were among the most common of
this type) even had a graphic of a beetle on them (and still do)! While the
ability to send repeated dots automatically was very useful for professional
morse code operators, these were also significantly trickier to use than the
older manual keyers, and it could take some practice to ensure one didn't
introduce extraneous dots into the code by holding the key down a fraction too
long. In the hands of an inexperienced operator, a Vibroplex ``bug'' on the line
could mean that a lot of garbled Morse would soon be coming your way.

Further, the term ``bug'' has long been used among radio technicians to describe
a device that converts electromagnetic field variations into acoustic signals.
It is used to trace radio interference and look for dangerous radio emissions.
Radio community usage derives from the roach-like shape of the first versions
used by 19th century physicists. The first versions consisted of a coil of wire
(roach body), with the two wire ends sticking out and bent back to nearly touch
forming a spark gap (roach antennae). The bug is to the radio technician what
the stethoscope is to the stereotype medical doctor. This sense is almost
certainly ancestral to modern use of ``bug'' for a covert monitoring device, but
may also have contributed to the use of ``bug'' for the effects of radio
interference itself.

Actually, use of ``bug''\opt{changes}{\footnote{Was single quotes, changed for
consistency.}} in the general sense of a disruptive event goes back to
Shakespeare! (Henry VI, part III --\opt{changes}{\footnote{Was a single hyphen,
but there is no grammatical construction that uses a single hyphen, changed to
an en dash.}} Act V, Scene II: King Edward: ``So, lie thou there. Die thou; and
die our fear; For Warwick was a bug that fear'd us all.'') In the first edition
of Samuel Johnson's dictionary one meaning of `bug' is ``A frightful object; a
walking spectre''; this is traced to `bugbear', a Welsh term for a variety of
mythological monster which (to complete the circle) has recently been
reintroduced into the popular lexicon through fantasy role-playing games.

In any case, in jargon the word almost never refers to insects. Here is a
plausible conversation that never actually happened:

``There is a bug in this ant farm!''

``What do you mean? I don't see any ants in it.''

``That's the bug.''

A careful discussion of the etymological issues can be found in a paper by Fred
R. Shapiro, 1987, ``Entomology of the Computer Bug: History and Folklore'',
American Speech 62(4):376-378.

[There has been a widespread myth that the original bug was moved to the
Smithsonian, and an earlier version of this entry so asserted. A correspondent
who thought to check discovered that the bug was not there.  While investigating
this in late 1990, your editor discovered that the NSWC still had the bug, but
had unsuccessfully tried to get the Smithsonian to accept it -- and that the
present curator of their History of American Technology Museum didn't know this
and agreed that it would make a worthwhile exhibit. It was moved to the
Smithsonian in mid-1991, but due to space and money constraints was not actually
exhibited years afterwards.  Thus, the process of investigating the
original-computer-bug bug fixed it in an entirely unexpected way, by making the
myth true! --ESR]

