\mainentry{BSD} /B-S-D/ n.

[abbreviation for `Berkeley Software Distribution'] a family of \citeentry{Unix}
versions for the \citeentry{DEC VAX} and PDP-11 developed by Bill Joy and others
at \citeentry{Berzerkeley} starting around 1980, incorporating paged virtual
memory, TCP/IP networking enhancements, and many other features. The BSD
versions (4.1, 4.2, and 4.3) and the commercial versions derived from them
(SunOS, ULTRIX, and Mt. Xinu) held the technical lead in the Unix world until
AT\&T's successful standardization efforts after about 1986; descendants are
still widely popular. Note that BSD versions going back to 2.9 are often
referred to by their version numbers, without the BSD prefix. See
\citeentry{4.2}, \citeentry{Unix}, \citeentry{USG Unix}.

