\mainentry{bogotify} /boh-go't*-fi:/ vt.

To make or become bogus. A program that has been changed so many times as to
become completely disorganized has become bogotified. If you tighten a nut too
hard and strip the threads on the bolt, the bolt has become bogotified and you
had better not use it any more. This coinage led to the notional
`autobogotiphobia' defined as `the fear of becoming bogotified'; but it is not
clear that the latter has been `live' jargon rather than a self-conscious joke
in jargon about jargon. See also \citeentry{bogosity}, \citeentry{bogus}.

