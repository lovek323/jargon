\mainentry{bob} n.

At Demon Internet, all tech support personnel
\opt{changes}{
    \footnote{Was `personal' instead of `personnel', an obvious misspelling.}
}
are called ``Bob''. (Female support personnel have an option on ``Bobette'').
This has nothing to do with Bob the divine drilling-equipment salesman of the
\citeentry{Church of the SubGenius}. Nor is it acronymized from ``Brother Of
\citeentry{BOFH}'', though all parties agree it could have been. Rather it was
triggered by an unusually large draft of new tech-support people in 1995. It was
observed that there would be much duplication of names. To ease the confusion,
it was decided that all support techs would henceforth be known as ``Bob'', and
identity badges were created labelled ``Bob 1'' and ``Bob 2''. (No, we never got
any further.)
\opt{changes}{
    \footnote{
        The full point was originally outside the closing parenthesis, which is
        both inconsistent and incorrect.
    }
}

The reason for ``Bob'' rather than anything else is due to a \citeentry{luser}
calling and asking to speak to ``Bob'', despite the fact that no ``Bob'' was
currently working for Tech Support. Since we all know ``the customer is always
right'', it was decided that there had to be at least one ``Bob'' on duty at all
times, just in case.

This silliness
\opt{changes}{\footnote{This was misspelt `sillyness'.}}
inexorably snowballed. Shift leaders and managers began to refer to their group
of ``bobs''. Whole ranks of support machines were set up (and still exist in the
DNS as of 1999) as bob1 through bobN. Then came alt.tech-support.recovery, and
it was filled with Demon support personnel. They all referred to themselves, and
to others, as `bob', and after a while it caught on. There is now a
Bob Code (\url{http://bob.bob.bofh.org/\%7Egiolla/bobcode.html}) describing the
Bob nature.

