\mainentry{bit} n.
[from the mainstream meaning and `Binary digIT']
\begin{inparaenum}
    \item {[}techspeak] The unit of information; the amount of information
        obtained by asking a yes-or-no question for which the two outcomes are
        equally probable.
    \item {[}techspeak] A computational quantity that can take on one of two
        values, such as true and false or 0 and 1.
    \item A mental flag: a reminder that something should be done eventually,
        ``I have a bit set for you.'' (I haven't seen you for a while, and I'm
        supposed to tell or ask you something.)
    \item More generally, a (possibly incorrect) mental state of belief. ``I
        have a bit set that says that you were the last guy to hack on EMACS.''
        (Meaning ``I think you were the last guy to hack on EMACS, and what I am
        about to say is predicated on this, so please stop me if this isn't
        true.'')
\end{inparaenum}

``I just need one bit from you'' is a polite way of indicating that you intend
only a short interruption for a question that can presumably be answered yes or
no.

A bit is said to be `set' if its value is true or 1, and `reset' or `clear' if
its value is false or 0. One speaks of setting and clearing bits. To
\citeentry{toggle} or `invert' a bit is to change it, either from 0 to 1 or from
1 to 0. See also \citeentry{flag}, \citeentry{trit}, \citeentry{mode bit}.

The term `bit' first appeared in print in the computer-science sense in 1949,
and seems to have been coined by early statistician and computer scientist John
Tukey. Tukey records that it evolved over a lunch table as a handier alternative
to `bigit' or `binit'.

