\mainentry{buzz} vi.

\begin{inparaenum}
\item Of a program, to run with no indication of progress and perhaps without
    guarantee of ever finishing; esp. said of programs thought to be executing
    tight loops of code. A program that is buzzing appears to be
    \citeentry{catatonic}, but never gets out of catatonia, while a buzzing loop
    may eventually end of its own accord. ``The program buzzes for about 10
    seconds trying to sort all the names into order.'' See \citeentry{spin}; see
    also \citeentry{grovel}.
\item {[}ETA Systems] To test a wire or printed circuit trace for continuity,
    esp. by applying an AC rather than DC signal. Some wire faults will pass DC
    tests but fail an AC buzz test.
\item To process an array or list in sequence, doing the same thing to each
    element. ``This loop buzzes through the tz array looking for a terminator
    type.''
\end{inparaenum}

