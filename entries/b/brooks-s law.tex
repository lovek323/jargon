\mainentry{Brooks's Law} prov.

``Adding manpower to a late software project makes it later'' -- a result of the
fact that the expected advantage from splitting development work among $N$
programmers is $O\left(N\right)$ (that is, proportional to $N$), but the
complexity and communications cost associated with coordinating and then merging
their work is $O\left(N^2\right)$ (that is, proportional to the square of $N$).
The quote is from Fred Brooks, a manager of IBM's OS/360 project and author of
``The Mythical Man-Month'' (Addison-Wesley, 1975, ISBN 0-201-00650-2), an
excellent early book on software engineering. The myth in question has been most
tersely expressed as ``Programmer time is fungible'' and Brooks established
conclusively that it is not. Hackers have never forgotten his advice (though
it's not the whole story; see \citeentry{bazaar}); too often,
\citeentry{management} still does. See also \citeentry{creationism},
\citeentry{second-system effect}, \citeentry{optimism}.

