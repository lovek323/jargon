\mainentry{buffer overflow} n.

What happens when you try to stuff more data into a buffer (holding area) than
it can handle?\opt{changes}{\footnote{Was `.', but `?' makes more sense.}} This
problem is commonly exploited by \citeentry{cracker}s to get arbitrary commands
executed by a program running with root permissions. This may be due to a
mismatch in the processing rates of the producing and consuming processes (see
\citeentry{overrun} and \citeentry{firehose syndrome}), or because the buffer is
simply too small to hold all the data that must accumulate before a piece of it
can be processed. For example, in a text-processing tool that
\citeentry{crunch}es a line at a time, a short line buffer can result in
\citeentry{lossage} as input from a long line overflows the buffer and trashes
data beyond it. The term is used of and by humans in a metaphorical sense.
``What time did I agree to meet you? My buffer must have overflowed.'' Or ``If I
answer that phone my buffer is going to overflow.'' See also \citeentry{spam},
\citeentry{overrun screw}.

