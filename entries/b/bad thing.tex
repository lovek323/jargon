\mainentry{Bad Thing} n.

[very common; from the 1930 Sellar \& Yeatman parody ``1066 And All That'']
Something that can't possibly result in an improvement of the subject. This term
is always capitalized, as in ``Replacing all of the 9600-baud modems with
bicycle couriers would be a Bad Thing''. Oppose \citeentry{Good Thing}. British
correspondents confirm that \citeentry{Bad Thing} and \citeentry{Good Thing}
(and prob. therefore \citeentry{Right Thing} and \citeentry{Wrong Thing}) come
from the book referenced in the etymology, which discusses rulers who were Good
Kings but Bad Things. This has apparently created a mainstream idiom on the
British side of the pond. It is very common among American hackers, but not in
mainstream usage here. Compare \citeentry{Bad and Wrong}.

