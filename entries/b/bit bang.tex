\mainentry{bit bang} n.

Transmission of data on a serial line, when accomplished by rapidly tweaking a
single output bit, in software, at the appropriate times. The technique is a
simple loop with eight OUT and SHIFT instruction pairs for each byte. Input is
more interesting. And full duplex (doing input and output at the same time) is
one way to separate the real hackers from the \citeentry{wannabee}s.

Bit bang was used on certain early models of Prime computers, presumably when
UARTs were too expensive, and on archaic 280 micros with a Zilog PIO but no SIO.
In an interesting instance of the \citeentry{cycle of reincarnation}, this
technique returned to use in the early 1990s on some RISC architectures because
it consumes such an infinitesimal part of the processor that it actually makes
sense not to have a UART. Compare \citeentry{cycle of reincarnation}.

