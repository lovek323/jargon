\mainentry{beta} /bay't*/, /be't*/ or (Commonwealth) /bee't*/ n.

\begin{inparaenum}
    \item Mostly working, but still under test; usu. used with `in': `in beta'.
        In the \citeentry{Real World}, system (hardware or software) often go
        through two stages of release testing: Alpha (in-house) and Beta
        (out-house?). Beta releases are generally made to a group of lucky (or
        unlucky) trusted customers.
    \item Anything that is new and experimental. ``His girlfriend is in beta''
        means that he is still testing for compatibility and reserving judgment.
    \item Flaky; dubious; suspect (since beta software is notoriously buggy).
\end{inparaenum}

Historical note: More formally, to beta-test is to test a pre-release
(potentially unreliable) version of a piece of software by making it available
to selected (or self-selected) customers and users. This term derives from early
1960s terminology for product cycle checkpoints, first used at IBM but later
standard throughout the industry. `Alpha Test' was the unit, module, or
component test phase; `Beta Test' was initial system test. These themselves came
from earlier A- and B- tests for hardware. The A-test was a feasibility and
manufacturability evaluation done before any commitment to design and
development. The B-test was a demonstration that the engineering model functions
as specified. The C-test (corresponding to today's beta) was the B-test
performed on early samples of the production design, and the D test was the C
test repeated after the model had been in production a while.

