\mainentry{back door} n.

[common] A hole in the security of a system deliberately left in place by
designers or maintainers. The motivation for such holes is not always sinister;
some operating systems, for example, come out of the box with privileged
accounts intended for use by field service technicians or the vendor's
maintenance programmers. Syn. \citeentry{trap door}; may also be called a
`wormhole'. See also \citeentry{iron box}, \citeentry{cracker},
\citeentry{worm}, \citeentry{logic bomb}.

Historically, back doors have often lurked in systems longer than anyone
expected or planned, and a few have become widely known. Ken Thompson's 1983
Turing Award lecture to the ACM admitted the existence of a back door in early
Unix versions that may have qualified as the most fiendishly clever security
hack of all time. In this scheme, the C compiler contained code that would
recognize when the `login' command was being recompiled and insert some code
recognizing a password chosen by Thompson, giving him entry to the system
whether or not an account had been created for him.

Normally such a back door could be removed by removing it from the source code
for the compiler and recompiling the compiler. But to recompile the compiler,
you have to use the compiler -- so Thompson also arranged that the compiler
would recognize when it was compiling a version of itself, and insert into the
recompiled compiler the code to insert into the recompiled `login' the code to
allow Thompson entry -- and, of course, the code to recognize itself and do the
whole thing again the next time around! And having done this once, he was then
able to recompile the compiler from the original sources; the hack perpetuated
itself invisibly, leaving the back door in place and active but with no trace in
the sources.

The talk that suggested this truly moby hack was published as ``Reflections on
Trusting Trust'', ``Communications of the ACM 27'', 8 (August 1984), pp.
761--763 (text available at \url{http://www.acm.org/classics}). Ken Thompson has
since confirmed that this hack was implemented and that the Trojan Horse code
did appear in the login binary of a Unix Support group machine. Ken says the
crocked compiler was never distributed. Your editor has heard two separate
reports that suggest that the crocked login did make it out of Bell Labs,
notably to BBN, and that it enabled at least one late-night login across the
network by someone using the login name `kt'.

