\mainentry{BBS} /B-B-S/ n.

[common; abbreviation, `Bulletin Board System'] An electronic bulletin board
system; that is, a message database where people can log in and leave broadcast
messages for others grouped (typically) into \citeentry{topic group}s. The term
was especially applied to the thousands of local BBS systems that operated
during the pre-Internet microcomputer era of roughly 1980 to 1995, typically run
by amateurs for fun out of their homes on MS-DOS boxes with a single modem line
each. Fans of Usenet and Internet or the big commercial timesharing bboards such
as CompServe, and GEnie tended to consider local BBSs\footnote{This was
originally, `BBSes', but I changed it to keep it in line with usage in the rest
of the book. ---JDO} the low-rent district of the hacker culture, but they
served a valuable function by knitting together lots of hackers and users in the
personal-micro world who would otherwise have been unable to exchange code at
all. Post-Internet, BBSs are likely to be local newsgroups on an ISP; efficiency
has increased but a certain flavor has been lost. See also \citeentry{bboard}.

