\mainentry{bogon} /boh'gon/ n.

[very common; by analogy with proton/electron/neutron, but doubtless reinforced
after 1980 by the similarity to Douglas Adams's `Vogons'; see the
Bibliography in Appendix C and note that Arthur Dent actually mispronounces
`Vogons' as `Bogons' at one point]
\begin{inparaenum}
\item The elementary particle of bogosity (see \citeentry{quantum
    bogodynamics}). For instance, ``the Ethernet is emitting bogons again''
    means that it is broken or acting in an erratic or bogus fashion.
\item A query packet sent from a TCP/IP domain resolvler to a root server,
    having the reply bit set instead of the query bit.
\item Any bogus or incorrectly formed packet sent on a network.
\item By synecdoche, used to refer to any bogus thing, as in ``I'd like to go to
    lunch with you but I've got to go to the weekly staff bogon''.
\item A person who is bogus or who says bogus things. This was historically the
    original usage, but has been overtaken by its derivative senses 1--4.
\end{inparaenum}
See also \citeentry{bogosity}, \citeentry{bogus}; compare \citeentry{psyton},
\citeentry{fat electrons}, \citeentry{magic smoke}.

The bogon has become the type case for a whole bestiary of nonce particle names,
including the `clutron' or `cluon' (indivisible particle of cluefulness,
obviously the antiparticle of the bogon) and the futon (elementary particle of
\citeentry{randomness}, or sometimes of lameness). These are not so much live
usages in themselves as examples of a live meta-usage: that is, it has become a
standard joke or linguistic maneuver to ``explain'' otherwise mysterious
circumstances by inventing nonce particle names. And these imply nonce particle
theories, with all their dignity or lack thereof (we might note parenthetically
that this is a generalization from ``(bogus particle) theories'' to ``bogus
(particle theories)''!). Perhaps such particles are the modern-day equivalents
of trolls and wood-nymphs as standard starting-points around which to construct
explanatory myths. Of course, playing on an existing word (as in the `futon')
yields additional flower. Compare \citeentry{magic smoke}.

