\mainentry{bitblt} /bit'blit/ n.

[FROM \citeentry{BLT}, q.v.]
\begin{inparaenum}
    \item {[}common] Any of a family of closely related algorithms for moving
        and copying rectangles of bits between main and display memory on a
        bit-mapped device, or between two areas of either main or display memory
        (the requirement to do the \citeentry{Right Thing} in the case of
        overlapping source and destination rectangles is what makes
        bitblt\opt{changes}{\footnote{This was originally `BitBlt', but I
        changed it to `bitblt' for consistency.}} tricky).
    \item Synonym for \citeentry{blit} or \citeentry{BLT}. Both uses are
        borderline techspeak.
\end{inparaenum}

