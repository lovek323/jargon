\mainentry{byte} /bi:t/ n.

[techspeak] A unit of memory or data equal to the amount used to represent one
character; on modern architectures this is usually 8 bits, but may be 9 on
36-bit machines. Some older architectures used `byte' for quantities of 6 or 7
bits, and the PDP-10 supported `bytes' that were actually bitfields of 1 to 36
bits! These usages are now obsolete, and even 9-bit bytes have become rare in
the general trend toward power-of-2 word sizes.

Historical note: The term was coined by Werner Buchholz in 1956 during the early
design phase for the IBM Stretch computer; originally it was described as 1 to 6
bits (typical I/O equipment of the period used 6-bit chunks of information). The
move to an 8-bit byte happened in late 1956, and this size was later adopted and
promulgated as a standard by the System/360. The word was coined by mutating the
word `bite' so it would not be accidentally misspelled as \citeentry{bit}. See
also \citeentry{nybble}.

