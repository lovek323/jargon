\mainentry{by hand} adv.

[common]
\begin{inparaenum}
\item Said of an operation (especially a repetitive, trivial, and/or tedious
    one) that ought to be performed automatically by the computer, but which a
    hacker instead has to step tediously through. ``My mailer doesn't have a
    command to include the text of the message I'm replying to, so I have to do
    it by hand.'' This does not necessarily mean the speaker has to retype a
    copy of the message; it might refer to, say, dropping into a subshell from
    the mailer, making a copy of one's mailbox file, reading that into an
    editor, locating the top and bottom of the message in question, deleting the
    rest of the file, inserting `$>$' characters on each line, writing the file,
    leaving the editor, returning to the mailer, reading the file in, and later
    remembering to delete the file. Compare \citeentry{eyeball search}.
\item By extension, writing code which does something in an explicit or
    low-level way for which a presupplied library routine ought to have been
    available. ``This cretinous B-tree library doesn't supply a decent iterator,
    so I'm having to walk the trees by hand.''
\end{inparaenum}

