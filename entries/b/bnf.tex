\end{multicols}

\hrule

\mainentry{BNF} /B-N-F/ n.

\begin{inparaenum}
\item {[}techspeak] Acronym for `Backus Normal Form' (often incorrectly expanded
    as `Backus-Naur Form'), a metasyntactic notion used to specify the syntax of
    programming languages, command sets, and the link. Widely used for language
    description but seldom documented anywhere, so that it must usually be
    learned by osmosis from other hackers. Consider this BNF for a U.S. postal
    address:

    \begin{verbatim}
<postal address> ::= <name-part> <street-address> <zip-part>

<personal-part> ::= <name> | <initial> "."

<name-part> ::= <personal-part> <last-name> [<jr-part>] <EOL>
              | <personal-part> <name-part>

<street-address> ::= [<apt>] <house-num> <street-name> <EOL>

<zip-part> ::= <town-name> "," <state-code> <ZIP-code> <EOL>
    \end{verbatim}

    This translates into English as: ``A postal-address consists of a name-part,
    followed by a street-address part, followed by a zip-code part. A
    personal-part consists of either a first name or an initial followed by a
    dot. A name-part consists of either: a personal-part followed by a last name
    followed by an optional `jr-part' (Jr., Sr., or dynastic number) and
    end-of-line, or a personal part followed by a name part (this rule
    illustrates the use of recursion in BNFs, covering the case of people who
    use street number, followed by a street name. A zip-part consists of a
    town-name, followed by a comma, followed by a state code, followed by a
    ZIP-code followed by an end of line.'' Note that many things (such as the
    format of a personal-part, apartment specifier, or ZIP-code) are left
    unspecified. These are presumed to be obvious from context or detailed
    somewhere nearby. See also \citeentry{parse}.
\item Any of a number of variants and extensions of BNF proper, possibly
    containing some or all of the \citeentry{regexp} wildcards such as * or +.
    In fact the examplel above isn't the pure form invented for the Algol-60
    report; it uses [], which was introduced a few years later in IBM's PL/I
    definition but is now universally recognized.
\item In \citeentry{science-fiction fandom}, a `Big-Name Fan' (someone famous or
    notorious). Years ago a fan started handing out black-on-green BNF buttons
    at SF conventions; this confused the hacker contingent terribly.
\end{inparaenum}

\ 

\hrule

\begin{multicols}{2}

