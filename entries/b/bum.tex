\mainentry{bum}

\begin{inparaenum}
\item vt. To make highly efficient, either in time or space, often at the expense
    of clarity. ``I managed to bum three more instructions out of that code.''
    ``I spent half the night bumming the interrupt code.'' In 1996, this term
    and the practice it describes are semi-obsolete. In \citeentry{elder days},
    John McCarthy (inventor of \citeentry{LISP}) used to compare some
    efficiency-obsessed hackers among his students to ``ski bums''; thus,
    optimization became ``program bumming'', and eventually just ``bumming''.
\item To squeeze out excess; to remove something in order to improve whatever it
    was removed from (without changing function; this distinguishes the process
    from a \citeentry{featurectomy}).
\item n. A small change to an algorithm, program, or hardware device to make it
    more efficient. ``This hardware bum makes the jump instruction faster.''
    Usage: now uncommon, largely superseded by v. \citeentry{tune} (and n.
    \citeentry{tweak}, \citeentry{hack}), though none of these exactly
    captures\opt{changes}{\footnote{Was `capture', changed to `captures' as
    `none' takes the singular.}} sense 2. All these uses are rare in
    Commonwealth hackish, because in the parent dialects of English the noun
    `bum' is a rude synonym for `buttocks' and the verb `bum' for buggery.
\end{inparaenum}

