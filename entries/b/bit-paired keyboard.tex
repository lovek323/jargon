\end{multicols}

\hrule

\mainentry{bit-paired keyboard} n., obs.

(alt. `bit-shift keyboard') A non-standard keyboard layout that seems to have
originated with the Teletype ASR-33 and remained common for several years on
early computer equipment. The ASR-33 was a mechanical device (see
\citeentry{EOU}), so the only way to generate the character codes from
keystrokes was by some physical linkage. The design of the ASR-33 assigned each
character key a basic pattern that could be modified by flipping bits if the
SHIFT or the CTRL key was pressed. In order to avoid making the thing even more
of a kludge than it already was, the design had to group characters that shared
the same basic bit pattern on one key.

Looking at the ASCII chart, we find:

\begin{tabular}{rrrrrrrrrrr}
             & \multicolumn{10}{l}{low bits}\\
    high bits& 0000& 0001& 0010& 0011& 0100& 0101& 0110& 0111& 1000& 1001\\
    010      &     &    !&    "&   \#&   \$&   \%&   \&&    '&    (&    )\\
    011      &    0&    1&    2&    3&    4&    5&    6&    7&    8&    9
\end{tabular}

This is why the characters !"\#\$\%\&'() appear where they do on a Teletype
(thankfully, they didn't use shift-0 for space). The Teletype Model 33 was
actually designed before ASCII existed, and was originally intended to use a
code that contained these two rows:

\resizebox{\textwidth}{!}{
    \begin{tabular}{rrrrrrrrrrrrrrrrr}
                 & \multicolumn{16}{l}{low bits}\\
        high bits&0000& 0001& 0010& 0011& 0100& 0101& 0110& 0111& 1000& 1001& 1010& 1011& 1100& 1101& 1110& 1111\\
        10       &   )&    !&  bel&   \#&   \$&   \%&  wru&   \&&    *&    (&    "&    :&    ?&   \_&    ,&    .\\
        11       &   0&    1&    2&    3&    4&    5&    6&    7&    8&    9&    '&    ;&    /&    -&  esc&  del
    \end{tabular}
}

The result would have been something closer to a normal keyboard. But as it
happened, Teletype had to use a lot of persuasion just to keep ASCII, and the
Model 33 keyboard, from looking like this instead:

\begin{tabular}{lllllllllllllllll}
     & !& "& ?& \$& '& \&& -& (& )& ;& :&   *&   /& ,& .&\\
    0& 1& 2& 3&  4& 5&  6& 7& 8& 9& +& ~& $<$& $>$& *& |
\end{tabular}

Teletype's was not the weirdest variant of the \citeentry{QWERTY} layout widely
seen, by the way; that prize should probably go to one of several (differing)
arrangements on IBM's even clunkier 026 and 029 card punches.

When electronic terminals became popular, in the early 1970s, there was no
agreement in the industry over how the keyboards should be laid out. Some
vendors opted to emulate the Teletype keyboard, while others used the
flexibility of electronic circuitry to make their product look like an office
typewriter. Either choice was supported by the ANSI computer keyboard standard,
X4.14-1971, which referred to the alternatives as `logical bit pairing' and
`typewriter pairing'. These alternatives became known as `bit-paired' and
`typewriter-paired' keyboards. To a hacker, the bit-paired keyboard seemed far
more logical -- and because most hackers in those had never learned to
touch-type, there was little pressure from the pioneering users to adapt
keyboards to the typewriter standard.

The doom of the bit-paired keyboard was the large-scale introduction of the
computer terminal into the normal office environment, where out-and-out
technophobes were expected to use the equipment. The `typewriter-paired'
standard became universal, X4.14 was superseded by X4.23-1982, `bit-paired'
hardware was quickly junked or relegated to dusty corners, and both terms passed
into disuse.

However, in countries without a long history of touch typing, the argument
against the bit-paired keyboard layout was weak or nonexistent. As a result, the
standard Japanese keyboard, used on PCs, Unix boxen etc. still has all of the
!"\#\$\%\&'()" characters above the numbers in the ASR-33 layout.

\hrule

\begin{multicols}{2}

