\mainentry{BCPL} // n.

[abbreviation, `Basic Combined Programming Language']\footnote{This was a close
parenthesis `)' not a close bracket `]', but I changed it for consistency.
---JDO} A programming language developed by Martin Richards in Cambridge in
1967. It is remarkable for its rich syntax, small size of compiler (it can be
run in 16k) and extreme portability. It reached break-even point at a very early
stage, and was the language in which the original \citeentry{hello world}
program was written. It has been ported to so many different systems that its
creator confesses to having lost count. It has only one data type (a machine
word) which can be used as an integer, a character, a floating point number, a
pointer, or almost anything else, depending on context. BCPL was a precursor of
C, which inherited some of its features.

