\mainentry{big-endian} adj.

[common; From Swift's \worktitle{Gulliver's Travels} via the famous paper ``On
Holy Wars and a Plea for Peace'' by Danny Cohen, USC/ISI IEN 137, dated April 1,
1980].
\begin{inparaenum}
    \item Describes a computer architecture in which, within a given
        multi-byte numeric representation, the most significant byte has
        the lowest address (the word is stored `big-end-first'). Most
        processors, including the IBM 370 family, the
        \citeentry{PDP-10}, the Motorola microprocessor families, and
        most of the various RISC designs are big-endian. Big-endian byte
        order is also sometimes called `network order'. See
        \citeentry{little-endian}, \citeentry{middle-endian},
        \citeentry{NUXI problem}, \citeentry{swab}.
    \item An \citeentry{Internet address} the wrong way round. Most of
        the world follows the Internet standard and writes email
        addresses starting with the name of the computer and ending up
        with the name of the country. In the U.K., the Joint Networking
        Team had decided to do it the other way round before the
        Internet domain standard was established. Most gateway sites
        have \citeentry{ad-hockery} in their mailers to handle this, but
        can still be confused. In particular, the address
        me@uk.ac.bris.pys.as could be interpreted in JANET's big-endian
        way as one in the U.K. (domain uk) or in the standard
        little-endian way as one in the domain as (American Samoa) on
        the opposite side of the world.
\end{inparaenum}

\begin{new}
    \begin{quote}
        `` `That the said Quinbus Flestrin, having brought the imperial fleet of
        Blefuscu into the royal port, and being afterwards commanded by his
        imperial majesty to seize all the other ships of the said empire of
        Blefuscu, and reduce that empire to a province, to be governed by a
        viceroy from hence, and to destroy and put to death, not only all the
        Big-endian exiles, but likewise all the people of that empire who would
        not immediately forsake the Big-endian heresy, he, the said Flestrin,
        like a false traitor against his most auspicious, serene, imperial
        majesty, did petition to be excused from the said service, upon pretence
        of unwillingness to force the consciences, or destroy the liberties and
        lives of an innocent people. \dots

        \begin{flushright}
            Jonathan Swift, \worktitle{Gulliver's Travels}
            \url{http://www.gutenberg.org/files/829/829-0.txt}
        \end{flushright}
    \end{quote}

    \begin{quote}
        The root of the conflict lies much deeper than that. It is the question
        of which bit should travel first, the bit from the little end of the
        word, or the bit from the big end of the word? The followers of the
        former approach are called the Little-Endians, and the followers of the
        latter are called the Big-Endians. The details of the holy war between
        the Little-Endians and the Big-Endians are documented in
        \worktitle{Gulliver's Travel} \dots

        \begin{flushright}
            Danny Cohen, \worktitle{On Holy Wars and a Plea for Peace (USC/ISI
            IEN 137), 1 April 1980 \url{http://www.ietf.org/rfc/ien/ien137.txt}}
        \end{flushright}
    \end{quote}
\end{new}

