\mainentry{burn-in period} n.

\begin{inparaenum}
\item A factory test designed to catch systems with \citeentry{marginal}
    components before they get out the door; the theory is that burn-in will
    protect customers by outwaiting the steepest part of the \citeentry{bathtub
    curve} (see \citeentry{infant mortality}).
\item A period of indeterminate length in which a person using a computer is so
    intensely involved in his project that he forgets basic needs such as food,
    drink, sleep, etc. Warning: Excessive burn-in can lead to burn-out. See
    \citeentry{hack mode}, \citeentry{larval stage}.
\end{inparaenum}

Historical note: the origin of ``burn-in'' (sense 1) is apparently the practice
of setting a new-model airplane's brakes on fire, then extinguishing the fire,
in order to make them hold better. This was done on the first version of the
U.S. spy-plane, the U-2.

