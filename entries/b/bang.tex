\mainentry{bang}

\begin{inparaenum}
    \item n. Common spoken name for ! (ASCII 0100001), especially when used in
        pronouncing a \citeentry{bang path} in spoken hackish. In
        \citeentry{elder days} this was considered a CMUish usage, with MIT and
        Stanford hackers preferring \citeentry{excl} or \citeentry{shriek}; but
        the spread of Unix has carried `bang' with it (esp. via the term
        \citeentry{bang path}) and it is now certainly the most common spoken
        name for !. Note that it is used exclusively for non-emphatic written !;
        one would not say ``Congratulations bang'' (except possibly for humorous
        purposes), but if one wanted to specify the exact characters `foo!' one
        would speak ``Eff oh oh bang''. See \citeentry{shriek},
        \citeentry{ASCII}.
    \item interj. An exclamation signifying roughly ``I have achieved
        enlightenment!'', or ``The dynamite has cleared out my brain!'' Often
        used to acknowledge that one has perpetrated a \citeentry{thinko}
        immediately after one has been called on it.
\end{inparaenum}

