\mainentry{bounce} v.

\begin{inparaenum}
\item {[}common; perhaps by analogy to a bouncing check] An electronic mail
    message that is undeliverable and returns an error notification to the
    sender is said to `bounce'. See also \citeentry{bounce message}.
\item {[}Stanford] To play volleyball. The now-demolished \citeentry{D. C. Power
    Lab} building used by the Stanford AI Lab in the 1970s had a volleyball
    court on the front lawn. From 5 P.M. to 7 P.M. was the scheduled maintenance
    time for the computer, so every afternoon at 5 would come over the intercom
    the cry: ``Now hear this: bounce, bounce!'', followed by Brian McCune loudly
    bouncing a volleyball on the floor outside the offices of known
    volleyballers.
\item To engage in sexual intercourse; prob. from the expression `bouncing the
    mattress', but influenced by Roo's psychosexually loaded ``Try bouncing me,
    Tigger!'' from the ``Winnie-the-Pooh'' books. Compare \citeentry{boink}.
\item To casually reboot a system in order to clear up a transient problem.
    Reported primarily among \citeentry{VMS} and \citeentry{Unix} users.
\item {[}VM/CMS programmers] Automatic warm-start of a machine after an error.
    ``I logged on this morning and found it had bounced 7 times during the
    night''.
\item {[}IBM] To \citeentry{power cycle} a peripheral in order to reset it.
\end{inparaenum}

