\mainentry{brute force} adj.

Describes a primitive programming style, one in which the programmer relies on
the computer's processing power instead of using his or her own intelligence to
simplify the problem, often ignoring problems of scale and applying naive
methods suited to small problems directly to large ones. The term can also be
used in reference to programming style: brute-force programs are written in a
heavyhanded, tedious way, full of repetition and devoid of any elegance or
useful abstraction (see also \citeentry{brute force and ignorance}).

The canonical example of a brute-force algorithm is associated with the
`traveling salesman problem' (TSP), a classical \citeentry{NP}-hard problem:
Suppose a person is in, say, Boston, and wishes to drive to $N$ other cities. In
what order should the cities be visited in order to minimize the distance
traveled\opt{changes}{\footnote{Was `travelled', but this is inconsistent with
the name of the problem, i.e., `traveling salesman problem'.}}? The brute-force
method is to simply generate all possible routes and compare the distances;
while guaranteed to work and simple to implement, this algorithm is clearly very
stupid in that it considers even obviously absurd routes (like going from Boston
to Houston via San Francisco and New York, in that order). For very small $N$ it
works well, but it rapidly becomes absurdly inefficient when $N$ increases (for
$N=15$, there are already 1,307,674,368,000 possible routes to consider, and for
$N=1000$ -- well, see \citeentry{bignum}). Sometimes, unfortunately, there is no
better general solution than brute force. See also \citeentry{NP-}.

A more simple-minded example of brute-force programming is finding the smallest
number in a large list by first using an existing program to sort the list in
ascending order, and then picking the first number off the front.

Whether brute-force programming should actually be considered stupid or not
depends on the context; if the problem is not terribly big, the extra CPU time
spent on a brute-force solution may cost less than the programmer time it would
take to develop a more `intelligent' algorithm. Additionally, a more intelligent
algorithm may imply more long-term complexity cost and bug-chasing than are
justified by the speed improvement.

Ken Thompson, co-inventor of Unix, is reported to have uttered the epigram
``When in doubt, use brute force''. He probably intended this as a \citeentry{ha
ha only serious}, but the original Unix kernel's preference for simple, robust,
and portable algorithms over \citeentry{brittle} `smart' ones does seem to be a
significant factor in the success of that OS. Like so many other tradeoffs in
software design, the choice between brute force and complex, finely-tuned
cleverness is often a difficult one that requires both engineering savvy and
delicate esthetic judgment.

