\mainentry{bare metal}\ 

\begin{inparaenum}
    \item {[}common] New computer hardware, unadorned with such snares and
        delusions as an \citeentry{operating system}, an \citeentry{HLL}, or
        even assembler. Commonly used in the phrase `programming on the bare
        metal', which refers to the arduous work of \citeentry{bit bashing}
        needed to create these basic tools for a new machine. Real bare-metal
        programming involves things like building boot roms and BIOS chips,
        implementing basic monitors used to test device drivers, and writing the
        assemblers that will be used to write the compiler back ends that will
        give the new machine a real development environment.
    \item `Programming on the bare metal' is also used to describe a style of
        \citeentry{hand-hacking} that relies on the bit-level peculiarities of a
        particular hardware design, esp. tricks for speed and space optimization
        that rely on crocks such as overlapping instructions (or, as in the
        famous case described in \citeentry{The Story of Mel} (in
        \citeappendix{A}), interleaving of opcodes on a magnetic drum to
        minimize fetch delays due to the device's rotational latency). This sort
        of thing has become less common as the relative costs of programming
        time and machine resources have changed, but is still found in heavily
        constrained environments such as industrial embedded systems, and in the
        code of hackers who just can't let go of that low-level control. See
        \citeentry{Real Programmer}.
\end{inparaenum}

In the world of personal computing, bare metal programming (especiall in sense 1
but sometimes also in sense 2) is often considered a \citeentry{Good Thing}, or
at least a necessary evil (because these machines often have been sufficiently
slow and poorly designed to make it necessary; see \citeentry{ill-behaved}).
There, the term usually refers to bypassing the BIOS or OS interface and writing
the application to directly access device registers and machine addresses. ``To
get 19.2 kilobaud on the serial port, you need to get down to the bare metal.''
People who can do this sort of thing well are held in high regard.

