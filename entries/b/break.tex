\mainentry{break}

\begin{inparaenum}
\item vt. To cause to be \citeentry{broken} (in any sense). ``Your latest patch
    to the editor broke the paragraph commands.''
\item v. (of a program) To stop temporarily, so that it may be debugged. The
    place where it stops is a `breakpoint'.
\item {[}techspeak] vi. To send an RS-232 break (two character widths of line
    high) over a serial comm line.
\item {[}Unix] vi. To strike whatever key currently causes the tty driver to
    send SIGINT to the current process. Normally, \citesense{break}{3}, delete
    or \citeentry{control-C} does this.
\item `break break' may be said to interrupt a conversation (this is an example
    of verb doubling). This usage comes from radio communications, which in turn
    probably came from landline telegraph/teleprinter usage, as badly abused in
    the Citizen's Band craze a few years ago.
\end{inparaenum}

