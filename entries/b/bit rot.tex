\mainentry{bit rot} n.

[common] Also \citeentry{bit decay}. Hypothetical disease the existence of which
has been deduced from the observation that unused programs or features will
often stop working after sufficient time has passed, even if `nothing has
changed'. The theory explains that bits decay as if they were radioactive. As
time passes, the contents of a file or the code in a program will become
increasingly garbled.

There actually are physical processes that produce such effects (alpha particles
generated by trace radionuclides in ceramic chip packages, for example, can
change the contents of computer memory \opt{changes}{\footnote{This was
originally `can change the contents of a computer memory`, but that didn't make
sense.}} unpredictably, and various kinds of subtle media failures can corrupt
files in mass storage), but they are quite rare (and computers are built with
error-detecting circuitry to compensate for them). The notion long favored among
hackers that cosmic rays are among the causes of such events turns out to be a
myth; see the \citeentry{cosmic rays} entry for details.

The term \citeentry{software rot} is almost synonymous. Software rot is the
effect, bit rot the notional cause.

