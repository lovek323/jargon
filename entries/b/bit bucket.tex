\mainentry{bit bucket} n.

[very common]
\begin{inparaenum}
    \item The universal data sink (originally, the mythical receptacle used to
        catch bits when they fall off the end of a register during a shift
        instruction). Discarded, lost, or destroyed data is said to have `gone
        to the bit bucket'. On \citeentry{Unix}, often used for
        \citeentry{/dev/null}. Sometimes amplified as `the Great Bit Bucket in
        the Sky'.
    \item The place where all lost mail and news messages eventually go. The
        selection is performed according to \citeentry{Finagle's Law}; important
        mail is much more likely to end up in the bit bucket than junk mail,
        which has an almost 100\% probability of getting delivered. Routing to
        the bit bucket is automatically performed by mail-transfer agents, news
        systems, and the lower layers of the network.
    \item The ideal location for all unwanted mail responses: ``Flames about
        this article to the bit bucket.'' Such a request is guaranteed to
        overflow one's mailbox with flames.
    \item Excuse for all mail that has not been sent. ``I mailed you those
        figures last week; they must have landed in the bit bucket.'' Compare
        \citeentry{black hole}.
\end{inparaenum}

This term is used purely in jest. It is based on the fanciful notion that bits
are objects that are not destroyed but only misplaced. This appears to have been
a mutation of an earlier term `bit box', about which the same legend was
current; old-time hackers also report that trainees used to be told that when
the CPU stored bits into memory it was actually pulling them `out of the bit
box'. See also \citeentry{chad box}.

Another variant of this legend has it that, as a consequence of the `parity
preservation law', the number of 1 bits that go to the bit bucket must equal the
number of 0 bits. Any imbalance results in bits filling up the bit bucket. A
qualified computer technician can empty a full bit bucket as part of scheduled
maintenance.

