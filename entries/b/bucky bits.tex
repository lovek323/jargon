\mainentry{bucky bits} /buh'kee bits/ n.

\begin{inparaenum}
\item obs. The bits produced by the CONTROL and META shift keys on a SAIL
    keyboard (octal 200 and 400 respectively), resulting in a 9-bit keyboard
    character set. The MIT AI TV (Knight) keyboards extended this with TOP and
    separate left and right CONTROL and META keys, resulting in a 12-bit
    character set; later, LISP Machines added such keys as SUPER, HYPER, and
    GREEK (see \citeentry{space-cadet keyboard}).
\item By extension, bits associated with `extra' shift keys on any keyboard,
    e.g., the ALT on an IBM PC or command and option keys on a Macintosh.
\end{inparaenum}

It has long been rumored that `bucky bits' were named for Buckminster Fuller
during a period when he was consulting at Stanford. Actually, bucky bits were
invented by Niklaus Wirth when he was at Stanford in 1964-65; he first suggested
the idea of an EDIT key to set the 8th bit of an otherwise 7-bit ASCII
character). It seems that, unknown to Wirth, certain Stanford hackers had
privately nicknamed him `Bucky' after a prominent portion of his dental anatomy,
and this nickname transferred to the bit. Bucky-bit commands were used in a
number of editors written at Stanford, including most notably TV-EDIT and NLS.

The term spread to MIT and CMU early and is now in general use. Ironically,
Wirth himself remained unaware of its derivation for nearly 30 years, until GLS
dug up this history in early 1993! See \citeentry{double bucky},
\citeentry{quadruple bucky}.

