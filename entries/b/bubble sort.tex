\mainentry{bubble sort} n.

Techspeak for a particular sorting technique in which pairs of adjacent values
in the list to be sorted are compared and interchanged if they are out of order;
thus, list entries `bubble upward' in the list until they bumb into one with a
lower sort value. Because it is not very good relative to other methods and is
the one typically stumbled on by \citeentry{naive} and untutored programmers,
hackers consider it the \citeentry{canonical} example of a naive algorithm.
(However, it's been shown by repeated experiment that below about
5,000\opt{changes}{\footnote{Was `5000', changed for consistency with other
large numbers.}} records bubble-sort is OK anyway.) The canonical example of a
really bad algorithm is \citeentry{bogo-sort}. A bubble sort might be used out
of ignorance, but any use of bogo-sort could issue only from brain damage or
willful perversity.

