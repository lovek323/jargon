\mainentry{Blue Screen of Death} n.

[common] This term is closely related to the older \citeentry{Black Screen of
Death} but much more common (many non-hackers have picked it up). Due to the
extreme fragility and bugginess of Microsoft Windows (3.1/95/NT versions),
misbehaving applications can crash the OS. The Blue Screen of Death, sometimes
decorated with hex error codes, is what you get when this happens. (Commonly
abbreviated \citeentry{BSOD}.) This event is sufficiently common to have
inspired the following haiku from Alan Tuplin:

\begin{verse}
    Your system which soared\\
    So freely on gliding wings\\
    now hangs, frozen and blue
\end{verse}

The following entry from the \citeentry{Salon Haiku Contest}, seems to have
predated popular use of the term (and may indeed have inspired it):

\begin{verse}
    Windows NT crashed.\\
    I am the Blue Screen of Death\\
    No one hears your screams.
\end{verse}

