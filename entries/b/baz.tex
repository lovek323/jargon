\mainentry{baz} /baz/ n.

\begin{inparaenum}
    \item {[}common] The third \citeentry{metasyntactic variable} ``Suppose we
        have three functions: FOO, BAR, and BAZ. FOO calls BAR, which calls
        BAZ\ldots'' (See also \citeentry{fum})
    \item interj. A term of mild annoyance. In this usage the term is often
        drawn out for 2 or 3 seconds, producing an effect not unlike the
        bleating of a sheep; /baaaaaaz/.
    \item Occasionally appended to \citeentry{foo} to produce `foobaz'.
\end{inparaenum}

Earlier versions of this lexicon derived `baz' as a Stanford corruption of
\citeentry{bar}. However, Pete Samson (compiler of the \citeentry{TMRC} lexicon)
reports it was already current when he joined TMRC in 1958. He says ``It came
from `Pogo'. Albert the Alligator, when vexed or outraged, would should `Bazz
Fazz!' or `Rowrbazzle!' The club layout was said to model the (mythical) New
England counties of Rowrfolk and Bassex (Rowrbazzle mingled with
Norfolk/Suffolk/Middlesex/Essex).''

