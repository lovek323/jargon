\mainentry{bottom-up implementation} n.

Hackish opposite of the techspeak term `top-down design'. It has been received
wisdom in most programming cultures that it is best to design from higher levels
of abstraction down to lower, specifying sequences of action in increasing
detail until you get to actual code. Hackers often find (especially in
exploratory designs that cannot be closely specified in advance) that it works
best to build things in the opposite order, by writing and testing a clean set
of primitive operations and then knitting them together. Naively applied, this
leads to hacked-together bottom-up implementations; a more sophisticated
response is `middle-out implementation', in which scratch code within primitives
at the mid-level of the system is gradually replaced with a more polished
version of the lowest level at the same time the structure above the midlevel is
being built.

