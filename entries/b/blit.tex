\mainentry{blit} /blit/ vt.

\begin{inparaenum}
\item {[}common] To copy a large array of bits from one part of a computer's
    memory to another part, particularly when the memory is being used to
    determine what is shown on a display screen. ``The storage allocator picks
    through the table and copies the good parts up into high memory, and then
    blits it all back down again.'' See \citeentry{bitblt}, \citeentry{BLT},
    \citeentry{dd}, \citeentry{cat}, \citeentry{blast}, \citeentry{snarf}. More
    generally, to perform some operation (such as toggling) on a large array of
    bits while moving them.
\item {[}historical, rare] Sometimes all-capitalized as `BLIT': an early
    experimental bit-mapped terminal designed by Rob Pike at Bell Labs, later
    commercialized as the AT\&T 5620. (The folk etymology from `Bell Labs
    Intelligent Terminal' is incorrect. Its creators liked to claim that the
    ``Blit'' stood for the Bacon, Lettuce, and Interactive Tomato.)
\end{inparaenum}

