\mainentry{baud} /bawd/ n.

[simplified from its technical meaning] n. Bits per second. Hence kilobaud or
Kbaud, thousands of bits per second. The technical meaning is `level transitions
per second'; this coincides with bps only for two-level modulation with no
framing or stop bits. Most hackers are aware of these nuances but blithely
ignore them.

Historical note: `baud' was originally a unit of telegraph signalling speed, set
at one pulse per second. It was proposed at the November, 1926 conference of the
Comit\'{e}e Consultatif International Des Communications T\'{e}l\'{e}graphiques
as an improvement on the then standard practice of referring to line speeds in
terms of words per minute, and named for Jean Maurice Emile Baudot (1845--1903),
a French engineer who did a lot of pioneering work in early teleprinters.

