\mainentry{background} n.,adj.,vt.

[common] To do a task `in background' is to do it whenever
\citeentry{foreground} matters are not claiming your undivided attention, and
`to background' something means to relegate it to a lower priority. ``For now,
we'll just print a list of nodes and links; I'm working on the graph-printing
problem in background.'' Note that this implies ongoing activity but at a
reduced level or in spare time, in contrast to mainstream `back burner' (which
connotes benign neglect until some future resumption of activity). Some people
prefer to use the term for processing that they have queued up for their
unconscious minds (a tack that one can often fruitfully take upon encountering
an obstacle in creative work), Compare \citeentry{amp off},
\citeentry{slopsucker}.

Technically, a task running in background is detached from the terminal where it
was started (and often running at a lower priority); oppose
\citeentry{foreground}. Nowadays this term is primarily associated with
\citeentry{Unix}, but it appears to have been first used in this sense on
OS/360.

