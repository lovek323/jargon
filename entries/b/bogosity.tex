\mainentry{bogosity} /boh-go's*-tee/ n.

\begin{inparaenum}
\item {[}origin. CMU, now very common] The degree to which something is
    \citeentry{bogus}. Bogosity is measured with a \citeentry{bogometer}; in a
    seminar, when a speaker says something bogus, a listener might raise his
    hand and say ``My bogometer just triggered''. More extremely, ``You just
    pinned my bogometer'' means you just said or did something so outrageously
    bogus that it is off the scale, pinning the bogometer needle at the highest
    possible reading (one might also say ``You just redlined my bogometer'').
    The agreed-upon units of bogosity are the \citeentry{microLenat}, and the
    \citeentry{microReid}.
    \opt{changes}{
        \footnote{
            This was originally ``\dots unit of bogosity is \dots`, but, since
            there seems to be more more than one accepted unit to measure
            bogosity, I have changed the pluralisation and conjunction.
        }
    }
\item The potential field generated by a \citeentry{bogon flux}; see
    \citeentry{quantum bogodynamics}.
\end{inparaenum}
See also \citeentry{bogon flux}, \citeentry{bogon filter}, \citeentry{bogus}.

