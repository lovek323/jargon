\mainentry{bang path} n.

[now historical] An old-style UUCP electronic-mail address specifying hops to
get from some assumed-reachable location to the addressee, so called because
each \citeentry{hop} is signified by a \citeentry{bang} sign. Thus, for example,
the path \texttt{...!bigsite!foovax!barbox!me} directs people to route their
mail to machine bigsite (presumably a well-known location accessible to
everybody) and from there through the machine foovax to the account of user me
on barbox.

In the bad old days of not so long ago, before autorouting mailers became
commonplace, people often published compound bang addresses using the
\texttt{\{ \}} convention (see \citeentry{glob}) to give paths from several big
machines, in the hopes that one's correspondent might be able to get mail to one
of them reliably (example: \texttt{...!\{seismo, ut-sally,
ihnp4\}!rice!beta!gamma!me}).  Bang paths of 8 to 10 hops were not uncommon in
1981. Late-night dial-up UUCP links would cause week-long transmission times.
Bang paths were often selected by both transmission time and reliability, as
messages would often get lost. See \citeentry{Internet address},
\citeentry{the network}, and \citeentry{sitename}.

