\mainentry{bot} n.

[common on IRC, MUD and among gamers; from `robot']
\begin{inparaenum}
\item An \citeentry{IRC} or \citeentry{MUD} user who is actually a program. On
    IRC, typically the robot provides some useful service. Examples are
    NickServ, which tries to prevent random users from adopting
    \citeentry{nick}s already claimed by others, and MsgServ, which allows one
    to send asynchronous messages to be delivered when the recipient signs on.
    Also common are `annoybots', such as KissServ, which perform no useful
    function except to send cute messages to other people. Service bots are less
    common on MUDs; but some others, such as the `Julia' bot active in 1990-91,
    have been remarkably impressive Turing-test experiments, able to pass as
    human for as long as ten or fifteen minutes of conversation.
\item An AI-controlled player in a computer game (especially a first-person
    shooter such as Quake) which, unlike ordinary monsters, operates like a
    human-controlled player, with access to a player's weapons and abilities. An
    example can be found at \url{http://www.telefragged.com/thefatal/}.
\end{inparaenum}

Note that bots in both senses were `robots' when the term first appeared in the
early 1990s, but the shortened form is now habitual.

