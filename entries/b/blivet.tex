\mainentry{blivet} /bliv'*t/ n.

[allegedly from a World War II military term meaning ``ten pounds of manure in a
five-pound bag'']
\begin{inparaenum}
\item An intractable problem.
\item A crucial piece of hardware that can't be fixed or replaced if it breaks.
\item A tool that has been hacked over by so many incompetent programmers that
    it has become an unmaintainable tissue of hacks.
\item An out-of-control but unkilllable development effort.
\item An embarrassing bug that pops up during a customer demo.
\item In the subjargon of computer security specialists, a denial-of-service
    attack performed by hogging limited resources that have no access controls
    (for example, shared spool space on a multi-user system).
\end{inparaenum}

This term has other meanings in other technical cultures; among experimental
physicists and hardware engineers of various kinds it seems to mean any random
object of unknown purpose (similar to hackish use of \citeentry{frob}). It has
also been used to describe an amusing trick-the-eye drawing resembling a
three-pronged fork that appears to depict a three-dimensional object until one
realizes that the parts fit together in an impossible way.

