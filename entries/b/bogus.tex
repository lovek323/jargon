\mainentry{bogus} adj.

\begin{inparaenum}
    \item Non-functional. ``Your patches are bogus.''
    \item Useless. ``OPCON is a bogus program.''
    \item False. ``Your arguments are bogus.''
    \item Incorrect. ``That algorithm is bogus.''
    \item Unbelievable. ``You claim to have solved the halting problem for
        Turing Machines? That's totally bogus.''
    \item Silly. ``Stop writing those bogus sagas.''
\end{inparaenum}

Astrology is bogus. So is a bolt that is obviously about to break. So is someone
who makes blatantly false claims to have solved a scientific problem. (This word
seems to have some, but not all, of the connotations of \citeentry{random} --
mostly the negative ones.)

It is claimed that `bogus' was originally used in the hackish sense at Princeton
in the late 1960s. It was spread to CMU and Yale by Michael Shamos, a migratory
Princeton alumnus. A glossary of bogus words was compiled at Yale when the word
was first popularized there about 1975-76. These coinages spread into hackerdom
from CMU and MIT. Most of them remained wordplay objects rather than actual
vocabulary items or live metaphors. Examples: `amboguous' (having multiple bogus
interpretations); `bogotissimo' (in a gloriously bogus manner); `bogotophile'
(one who is pathologically fascinated by the bogus); `paleobogology' (the study
of primeval bogosity).

Some bogowords, however, obtained sufficient live currency to be listed
elsewhere in this lexicon; see \citeentry{bogometer}, \citeentry{bogon},
\citeentry{bogotify}, and \citeentry{quantum bogodynamics} and the related but
unlisted \citeentry{Dr. Fred Mbogo}.

By the early 1980s `bogus' was also current in something like hacker usage sense
in West Coast teen slang, and it had gone mainstream by 1985. A correspondent
from Cambridge reports, by contrast, that these units of `bogus' grate on
British nerves; in Britain the word means, rather specifically, `counterfeit',
as in ``a bogus 10-pound note''.

