\mainentry{bazaar} n.,adj.

In 1997, after meditating on the success of \citeentry{GNU/Linux} for three
years, the Jargon File's former editor ESR wrote an analytical paper on hacker
culture and development models titled \citeentry{The Cathedral and the Bazaar}.
The main argument of the paper was that \citeentry{Brook's Law} is not the whole
story; given the right social machinery, debugging can be efficiently
parallelized across large numbers of programmers. The title metaphor caught on
(see also \citeentry{cathedral}), and the style of development typical in the
GNU/Linux community is now often referred to as the bazaar mode. Its
characteristics include releasing code early and often, and actively seeking the
largest possible pool of peer reviewers..

