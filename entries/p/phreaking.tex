\mainentry{phreaking} /freek'ing/ n.

[from `phone phreak']
\begin{inparaenum}
	\item The art and science of \citeentry{cracking} the phone network (so as, for example, to make free long-distance calls).
	\item By extension, security-cracking in any other context (especially, but not exclusively, on communications networks) see
		(\citeentry{cracking}).
\end{inparaenum}

At one time phreaking was a semi-respectable activity among hackers; there was a gentleman's agreement that phreaking as an intellectual
game and a form of exploration was OK, but serious theft of services was taboo. There was significant crossover between the hacker
community and the hard-core phone phreaks who ran semi-underground networks of their own through such media as the legendary
\worktitle{TAP Newsletter}. This ethos began to break down in the mid-1980s as wider dissemination of the techniques put them in the hands
of less responsible phreaks. Around the same time, changes in the phone network made old-style technical ingenuity less effective as a way
of hacking it, so phreaking came to depend more on overtly criminal acts such as stealing phone-card numbers. The crimes and punishments of
gangs like the `414 group' turned that game very ugly. A few old-time hackers still phreak casually just to keep their hand in, but most
these days have hardly even heard of `blue boxes' or any of the other paraphernalia of the greatest phreaks of yore.

\begin{new}
	The \worktitle{TAP Newsletter} seems to have previously been called the \worktitle{YIPL} or the
	\worktitle{Youth International Party Line Newsletter}. The first issue, accessible at
	\url{<http://artofhacking.com/tap/yipl/live/aoh_yipl01.htm>}, was published in June 1971.

	This issue (which includes a great deal of anti-Vietnam-War rhetoric) refers to its readers as ``phreakers''. In one of the letters to
	the editor, it is rendered ``phreeks''.

	The name was changed from \worktitle{YIPL} to \worktitle{TAP} (\worktitle{Technological American Party}) with issue 21, dated
	August--September 1973 (accessible at \url{<http://artofhacking.com/tap/tap/live/aoh_tap21.htm>}) which included the following
	explanation:

	\begin{quote}
		No fancy excuses: We changed our name because we want people to know where we really are and what we hope to become. Technological
		American Party is rapidly becoming a people's warehouse of technological information, and a name like Youth International Party
		Line simply didn't ring a bell, even if you were trying to find out how to contact the phone phreaks, except of course for the
		Party Line. We've been receiving so much information lately about gas and electric meters, locks, even chemistry, that a name
		change is definitely in order. We seriously doubt that phones will cease to be our main interest, but it really isn't fair to
		anyone to ignore the rest of what science has to offer us.
	\end{quote}
\end{new}

