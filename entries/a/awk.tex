\mainentry{awk} /awk/

\begin{inparaenum}
	\item n. [Unix techspeak] An interpreted language for massaging text data developed by Alfred Aho, Peter Weinberger, and Brian Kernighan
		(the name derives from their initials). It si characterized by C-like syntax, a declaration-free approach to variable typing and
		declarations, associative arrays, and field-oriented text processing. See also \citeentry{Perl}
	\item n. Editing term for an expression awkward to manipulate through normal \citeentry{regexp} facilities (for example, one containing
		a \citeentry{newline}).
	\item vt. To process data using awk(1).
\end{inparaenum}

