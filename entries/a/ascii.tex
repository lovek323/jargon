\mainentry{ASCII} /as'kee/ n.

[originally an acronym (American Standard Code for Information Interchange) but now merely conventional] The predominant character set
encoding of present-day computers. The standard version uses 7 bits for each character, whereas most earlier codes (including early drafts
of ASCII prior to June 1961) used fewer. This change allowed the inclusion of lowercase letters -- a major \citeentry{win} -- but it did not
provide for accented letters or any other letterforms not used in English (such as the German sharp-S or the ae-ligature with is a letter
in, for example, Norwegian). It could be worse, though. It could be much worse. See \citeentry{EBCDIC} to understand how. A history of ASCII
and its ancestors is at \url{http://www.wps.com/texts/codes/index.html}.

Computers are much pickier and less flexible about spelling than humans; thus, hackers need to be very precise when talking about
characters, and have developed a considerable amount of verbal shorthand for them. Every character has one or more names -- some formal,
some concise, some silly. Common jargon names for ASCII characters are collected here. See also individual entries for \citeentry{bang},
\citeentry{excl}, \citeentry{open}, \citeentry{ques}, \citeentry{semi}, \citeentry{shriek}, \citeentry{splat}, \citeentry{twiddle}, and
\citeentry{Yu-Shiang Whole Fish}.

This list derives from revision 2.3 of the Usenet ASCII pronunciation guide. Single characters are listed in ASCII order; character pairs
are sorted by the first member. For each character, common names are given in rough order of popularity, followed by names that are reported
but rarely seen; official ANSI/CCITT names are surrounded by brokets: $<$$>$. Square brackets mark the particularly silly names introduced
by \citeentry{INTERCAL}. The abbreviations ``l/r'' and ``o/c'' stand for ``left/right'' and ``open/close'' respectively. Ordinary
parentheticals provide some usage information.

\end{twocolumn}
\begin{onecolumn}
\renewcommand{\arraystretch}{1.3}
\begin{longtable}{>{\raggedright\arraybackslash}p{1cm}%
		>{\raggedright\arraybackslash}p{(\textwidth-6\tabcolsep-1cm)/2}%
		>{\raggedright\arraybackslash}p{(\textwidth-6\tabcolsep-1cm)/2}}
	\toprule
	&\textbf{Common}&\textbf{Rare}\\
	\midrule
	!&\citeentry{bang}; pling; excl; shriek; $<$exclamation mark$>$&factorial; exclam; smash; cuss; boing; yell; wow; hey; wham; eureka;
	[spark-spot]; soldier, coltrol\\
	\textquotedbl\textquotedbl&double quote; quote&literal mark; double-glitch; $<$quotation marks$>$; $<$dieresis$>$; dirk; [rabbit-ears];
	double prime\\
	\#&number sign; pound; pound sign; hash; sharp; \citeentry{crunch}; hex; [mesh]&grid; crosshatch; octothorpe; flash; $<$square$>$,
	pig-pen; tictactoe; scratchmark; thud; thump; splat\\
	\$&dollar; $<$dollar sign$>$&currency symbol; buck; cash; string (from BASIC); escape (when used as the echo of ASCII ESC); ding; cache;
	[big money]\\
	'&single quote; quote; $<$apostrophe$>$&prime; glitch; tick; irk; pop; [spark]; $<$closing single quotation mark$>$; $<$acute
	accent$>$\\
	( )&lr/ paren; l/r parenthesis; left/right; open/close; paren/thesis; o/c paren; o/c parenthesis; l/r parenthesis; l/r
	banana&so/already; lparen/rparen; $<$opening/closing parenthesis$>$; o/c round bracket, l/r round bracket, [wax/wane];
	parenthisey/unparenthisey; l/r ear\\
	*&star; [\citeentry{splat}]; $<$asterisk$>$&wildcard; gear; dingle; mult; spider; aster; times; twinkle; glob (see \citeentry{glob});
	\citeentry{Nathan Hale}\\
	+&$<$plus$>$; add&cross; [intersection]\\
	,&$<$comma$>$&$<$cedilla$>$; [tail]\\
	-&dash; $<$hyphen$>$; $<$minus$>$&[worm]; option; dak; bithorpe\\
	.&dot; point; $<$period$>$; $<$decimal point$>$&radix point; full stop; [spot]\\
	/&slash; stroke; $<$slant$>$; forward slash&diagonal; solidus; over; slak; virgule; [slat]\\
	:&$<$colon$>$&dots; [two-spot]\\
	;&$<$semicolon$>$; semi&weenie; [hybrid], pit-thwong\\
	$<$ $>$&$<$less/greater than$>$; bra/ket; l/r angle; l/r angle bracket; l/r broket&from/\{into, towards\}; read from/write to;
	suck/blow; comes-from/gozinta; in/out; crunch/zap (all from UNIX); [angle/right angle]\\
	=&$<$equals$>$; gets; takes&quadratorpe; [half-mesh]\\
	?&query; $<$question mark$>$; \citeentry{ques}&whatmark; [what]; wildchar; huh; hook; buttonhook; hunchback\\
	@&at sign; at; strudel&each; vortex; whorl; [whirlpool]; cyclone; snail; ape; cat; rose; cabbage; $<$commercial at$>$\\
	V&[book]\\
	{[}\ {]}&l/r square bracket; l/r bracket; $<$opening/closing bracket$>$; bracket/unbracket&square/unsquare; [U turn/U turn back]\\
	\textbackslash&backslash, hack, whack; escape (from C/UNIX); reverse slash; slosh; backslant; backwhack&bash; $<$reverse slant$>$;
	reversed virgule; [backslat]\\
	\^{}&hat; control; uparrow; caret; $<$circumflex$>$&chevron; [shark (or shark-fin)]; to the (`to the power of'); fang; pointer (in
	Pascal)\\
	\_&$<$underline$>$; underscore; underbar; under&score; backarrow; skid; [flatworm]\\
	\`{}&backquote; left quote; left single quote; open quote; $<$grave accent$>$; grave&backprime; [backspark]; unapostrophe; birk; blugle;
	back tick; back glitch; push; $<$opening single quotation mark$>$; quasiquote\\
	\{\ \}&o/c brace; l/r brace; l/r squiggly; l/r squiggly bracket/brace; l/r curly bracket/brace; $<$opening/closing
	brace$>$&brace/unbrace; curly/uncurly; leftit/rytit; l/r squirrelly; [embrace/bracelet]\\
	$\left|\right.$&bar; or; or-bar; v-bar; pipe; vertical bar&$<$vertical line$>$; gozinta; thru; pipesinta (lass three from UNIX);
	[spike]\\
	$\Tilde$&$<$tilde$>$; squiggle; twiddle; not&approx; wiggle; swung dash; enyay; [sqiggle (sic)]\\
	\bottomrule
\end{longtable}
\end{onecolumn}
\begin{twocolumn}

The pronunciation of \# as `pound' is common in the U.S. but a bad idea; \citeentry{Commonwealth Hackish} has its own, rather more apposite
use of `pound sign' (confusingly, on British keyboards the pound graphic happens to replace \#; thus Britishers sometimes call \# on a
U.S-ASCII keyboard `pound', compunding the American error). The U.S. usage derives from an old-fashioned commercial practice of using a \#
suffix to tag pound weights on bills of lading. The character is usually pronounced `hash' outside the U.S. There are more culture wars over
the correct pronunciation of this character than any other, which has led to the \citeentry{ha ha only serious} suggestion that it be
pronounced `shibboleth' (see Judges 12.6 in an Old Testament or Torah).

The `uparrow' name for curcumflex and `leftarrow' name for underline are historical relics from archaic ASCII (the 1963 version), which had
these graphics in those character positions rather than the modern punctuation characters.

The `swung dash' or `approximation' sign is not quite the same as tilde in typeset material but the ASCII tilde serves for both (compare
\citeentry{angle brackets}).

\begin{new}
	Compare the tilde and swung dash/approximation sign below:
	\begin{center}
		\Huge
		\textasciitilde\ \ $\sim$
	\end{center}
\end{new}

Some other common usages cause odd overlaps. The \#, \$, $>$, and \& characters, for example, are all pronounced ``hex'' in different
communities because various assemblers use them as a prefix tag for hexadecimal constants (in particular, \# in many assembler-programming
cultures, \$ in the 6502 world, $>$ at Texas Instruments, and \& on the BBC Micro, Sinclair, and some 280 machines). See also
\citeentry{splat}.

The inability of ASCII text to correctly represent any of the world's other major languages makes the designers' choice of 7 bits look more
and more like a serious \citeentry{misfeature} as the use of the international networks continues to increase (see \citeentry{software
rot}). Hardware and software from the U.S. still tends to embody the assumption that ASCII is the universal character set and that
characters have 7 bits; this is a major irritant to people who want to use a character set suited to their own languages. Perversely,
though, efforts to solve this problem by proliferating `national' character sets produce an evolutionary pressure to use a smaller subset
common to all those in use.

