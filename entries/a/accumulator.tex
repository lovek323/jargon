\mainentry{accumulator} n. obs.

\begin{inparaenum}
	\item Archaic term for a register. On-line use of it as a synonym for `register' is a fairly reliable indication that the user has been
		around for quite a while and/or that the architecture under discussion is quite old. The term in full is almost never used of
		microprocessor registers, for example, though symbolic names for arithmetic registers beginning in `A' derive from historical use of
		the term `accumulator' (and not, actually, from `arithmetic'). Confusingly, though, an `A' register name prefix may also stand for
		`address', as for example on the Motorola 680x0 family.
	\item A register being used for arithmetic or logic (as opposed to addressing or a loop index), especially one being used to accumulate
		a sum or count of many items. This use is in context of a particular routine or stretch of code. ``The FOOBAZ routines uses A3 as an
		accumulator.''
	\item One's in-basket (esp. among old-timers who might use sense 1). ``You want this reviewed? Sure, just put it in the accumulator.''
		(See \citeentry{stack}).
\end{inparaenum}

