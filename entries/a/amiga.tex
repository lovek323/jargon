\mainentry{Amiga} n.

A series of personal computer models originally sold by Commodore, based on 680x0 processors, custom support chips and an operating system
that combined some of the best features of Macintosh and Unix with compatibility with neither.

The Amiga was released just as the personal computing world standardized on IBM-PC clones. This prevented it from gaining serious market
share, despite the fact that the first Amigans had a substantial technological lead on the IBM XTs of the time. Instead, it acquired a small
but zealous population of enthusiastic hackers who dreamt of one day unseating the clones (see \citeentry{Amiga Persecution Complex}). The
traits of this culture are both spoofed and illuminated in \citeentry{The BLAZE Humor Viewer}. The strength of the Amiga platform seeded a
small industry building software and hardware for the platform, especially in graphics and video applications (see \citeentry{video
toaster}).

Due to spectacular mismanagement, Commodore did hardly any R\&D, allowing the competition to close Amiga's technological lead. After
Commodore went bankrupt in 1994 the technology passed through several hands, none of whom did much with it. However, the Amiga is still
being produced in Europe under license and has a substantial number of fans, which will probably extend the platform's life considerably.

