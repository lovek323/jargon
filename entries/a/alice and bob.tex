\mainentry{Alice and Bob} n.

The archetypal individuals used as examples in discussions of cryptographic
protocols. Originally, theorists would say something like: ``A communicates
with someone who claims to be B, so to be sure, A tests that B knows a secret
number K. So A sends to B a random number X. B then forms Y by encrypting X
under key K and sends Y back to A''. Because this sort of thing is quite hard
to follow, theorists stopped using the unadorned letters A and B to represent
the main players and started calling them Alice and Bob. So now we say ``Alice
communicates with someone claiming to be Bob, so to be sure, Alice tests that
Bob knows a secret number K. So Alice sends to Bob a random number X. Bob then
forms Y by encrypting X under key K and sends Y back to Alice''. A whole
mythology rapidly grew up around Alice and Bob; see
\url{http://www.conceptlabs.co.uk/alicebob.html}.

In Bruce Schneier's definitive introductory text \worktitle{Aplied
Cryptography} (2nd ed., 1996, John Wiley \& Sons, ISBN 0-471-11709-9) he
introduces a tableo f dramatis personae headed by Alice and Bob. Others include
Carol (a participant in three- and four-party protocols), Dave (a participant
in four-party protocols), Eve (an eavesdropper), Mallory (a malicious active
attacker), Trent (a trusted arbitrator), Walter (a warden), Peggy (a prover)
and Victor (a verifier). These names for roles are either already standard or,
given the wide popularity of the book, may be expected to quickly become so.  

