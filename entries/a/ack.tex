\mainentry{ACK} /ak/ interj.

\begin{inparaenum}
	\item {[}common; from the ASCII mnemonic for 0000110] Acknowledge. Used to register one's presence (compare mainstream Yo!). An
		appropriate response to \citeentry{ping} or \citeentry{ENQ}.
	\item {[}from the comic strip ``Bloom County''] An exclamatino of surprised disgust, esp. in ``Ack pffft!'' Semi-humorous. Generally
		this sense is not spelled in caps (ACK) and is distinguished by a following exclamation point.
	\item Used to politely interrupt someone to tell them you understand their point (see \citeentry{NAK}). Thus, for example, you might cut
		off an overly long explanation wit ``Ack. Ack. Ack. I get it now''.
\end{inparaenum}

There is also a usage ``ACK?'' (from sense 1) meaning ``Are you there?'', often used in email when earlier mail has produced no reply, or
during a lull in \citeentry{talk mode} to see if the person has gone away (the standard humorous response is of course \citesense{NAK}{2},
i.e., ``I'm not here'').

