\mainentry{alt} /awlt/

\begin{inparaenum}
	\item n. The alt shift key on an IBM PC or \citeentry{clone} keyboard; see \citesense{bucky bits}{2} (though typical PC usage does not
		simply set the 0200 bit).
	\item n. The `option' key on a Macintosh; use of this term usually reveals the speaker hacked PCs before coming to the Mac (see also
		\citeentry{feature key}, which is sometimes incorrectly called `alt').
	\item n., obs. [PDP-10; often capitalized to ALT] Alternate name for the ASCII ESC character (ASCII 0011011), after the keycap labeling
		on some older terminals; also `altmode' (/awlt'mohd/). This character was almost never pronounced `escape' on an ITS system, in
		\citeentry{TECO}, or under TOPS-10 -- always alt, as in ``Type alt alt to end a TECO command'' or ``alt-U onto the system'' (for
		``log onto the [ITS] system''). This usage probably arose because alt is more convenient to say than `escape', especially when
		followed by another alt or a character (or another alt and a character, for that matter).
	\item The alt hierarchy on Usenet, the tree of newsgroups created by users without a formal vote and approval procedure. There is amyth,
		not entirely implausible, that alt is acronymic for ``anarchists, lunatics, and terrorists''; but in fact it is simply short for
		``alternative''.
\end{inparaenum}

