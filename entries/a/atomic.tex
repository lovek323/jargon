\mainentry{atomic} adj.

[from Gk. `atomos', indivisible]
\begin{inparaenum}
	\item Indivisible; cannot be split up. For example, an instruction may be
		said to do several things `atomically', i.e., all the things are done
		immediately, and there is no chance of the instruction being
		half-completed or of another being interspersed. Used esp. to convey
		that an operation cannot be screwed up by interrupts. ``This routine
		locks the file and increments the file's semaphore atomically.''

	\item {[}primarily techspeak] Guaranteed to complete successfully or not at
		all, usu. refers to database transactions. If an error prevents a
		partially-performed transaction from proceeding to completion, it must
		be ``backed out,'' as the database must not be left in an inconsistent
		state.
\end{inparaenum}

Computer usage, in either of the above senses, has none of the connotations
that `atomic' has in mainstream English (i.e. of particles of matter, nuclear
explosions etc.).

