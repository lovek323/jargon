\mainentry{AI-complete} /A-I k*m-pleet'/ adj.

[MIT, Stanford: by analogy with `NP-complete' (see \citeentry{NP--})] Used to describe problems or subproblems in AI, to indicate that the
solution presupposes a solution to the `strong AI problem' (that is, the synthesis of a human-level intelligence). A problem that is
AI-complete is, in other words, just too hard.

Examples of AI-complete problems are `The Vision Problem' (building a system that can see as well as a human) and `The Natural Language
Problem' (building a system that can understand and speak a natural language as well as a human). These may appear to be modular, but all
attempts so far have foundered on the amount of context information and `intelligence' they seem to require. See also \citeentry{gedanken}.

