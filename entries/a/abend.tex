\mainentry{ABEND} /a'bend/, /*-bend'/ n.

[ABnormal END]
\begin{inparaenum}
	\item Abnormal termination (of software); \citeentry{crash}; \citeentry{lossage}. Derives from an error message on the IBM 360;
		used jokingly by hackers but seriously mainly by \citeentry{code grinder}s. Usually capitalized, but may appear as `abend'.
		Hackers will try to persuade you that ABEND is called `abend' because it is what system operators do to the machine late on Friday
		night when the want to call it a day, and hence is from the German `Abend' = `Evening'.
	\item {[}alt.alt.callahans] Absent By Enforced Net Deprivation -- used in the subject lines of postings warning friends of an
		imminent loss of Internet access. (This can be because of computer downtime, loss of provider, moving or illness.) Variants of this
		also appear: ABVND = `Absent By Voluntary Net Deprivation' and ABSEND = `Absent By Self-Enforced Net Deprivation' have been
		sighted.
\end{inparaenum}

\begin{new}
	\begin{usenet}
		We have the IBM 370 C compiler written by Bell Labs. When running under "MVS" it works fine. We wrote and compiled a program under
		"MVS". This took a second to compile, link and run \dots\ Run it under "MVT" and it loops and then abends after 10
		seconds or so

		\citeusenet{Atkinson, Brian}{unknown}{1983}{IBM C Compiler}{net.general}{xKFDtzP1lcc}{28 October}
	\end{usenet}

	\begin{usenet}
		As I said in my response to Douglas Reay, having realized that I've offended people through my deliberate obnoxiousness, I am going
		to take myself out of the situation. If I were less of a hypocrite, I'd resolve to stay away forever, but I don't have the strength
		of character to do that. Instead, I'll resolve to remain silent at least through the end of August, except to post the munich
		announcements.

		\citeusenet{Miles, Janet}{jmiles@usit.net}{1999}{Apology and ABVND [Was: Littleton backlash]}{soc.subculture.bondage-bdsm}{pZECvacA29A/LCk2bGyljsEJ}{30 April}
	\end{usenet}

	\begin{usenet}
		I have just seen the character demo of TRITON's new game "Into the Shadows". Now I understand why Triton has been absent so long. I
		am not that much into role playing games, so I don't know what the standards are today, but the movement and animation of the
		characters in the demo are fantastic. That's what I call realistic. So Triton spent their time well the last, well, two years. For
		all of you who haven't seen it yet: Don't waste any more time and download it immediately.

		\citeusenet{Liebermann, David}{deel@eikon.e-technik.tu-muenchen.de}{1995}{Into the Shadows demo by TRITON !!!}{comp.sys.ibm.pc.demos}{S32l9JJpeIg/etuN9xpPuxIJ}{27 November}
	\end{usenet}
\end{new}

