\mainentry{ANSI} /an'see/

\begin{inparaenum}
	\item n. [techspeak] The American Standards Institute. ANSI, along with the International Organisation for Standards (ISO), standardized
		the C programming language (see \citeentry{K\&R}, \citeentry{Classic C}), and promulgates many other important software standards.
	\item n. [techspeak] A terminal may be said to be `ANSI' if it meets the ANSI X.364 standard for terminal control. Unfortunately, this
		standard was both over-complicated and too permissive. It has been retired and replaced by the ECMA-48 standard, which shares both
		flaws.
	\item n. [BBS jargon] The set of screen-painting codes that most MS-DOS and Amiga computers accept. This comes from the ANSI.SYS device
		driver that must be loaded on an MS-DOS computer to view such codes. Unfortunately, neither DOS ANSI nor the BBS ANSIs derived from
		it exactly match the ANSI X.364 terminal standard. For example, the ESC-[1m code turns on the bold highlight on large machines, but
		in IBM PC/MS-DOS ANSI, it turns on `intense' (bright) colors. Also, in BBS-land, the term `ANSI' is often used to imply that a
		particular computer uses or can emulate the IBM high-half character set from MS-DOS. Particular use depends on context.
		Occasionally, the vanilla ASCII character set is used with the color codes, but on BBSs, ANSI and `IBM characters' tend to go
		together.
\end{inparaenum}

