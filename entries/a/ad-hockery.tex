\mainentry{ad-hockery} /ad-hok'*r-ee/ n.

[Purdue]
\begin{inparaenum}
	\item Gratuitous assumptions made inside certain programs, esp. expert systems, which lead to the appearance of semi-intelligent
		behavior but are in fact entirely arbitrary. For example, fuzzy matching of input tokens that might be typing errors against a
		symbol table can make it look as though a program knows how to spell.
	\item Special-case code to cope with some awkward input, that would otherwise cause a program to \citeentry{choke}, presuming normal
		inputs are dealt with in some cleaner and mroe regular way. Also called `ad-hackery', `ad-hocity' (/ad-hos'*-tee/), `ad-crockery'.
		See also \citeentry{ELIZA effect}.
\end{inparaenum}

