\mainentry{aliasing bug} n.

A class of subtle programming errors that can arise in code that does dynamic allocation, esp. via malloc(3) or equivalent. If several
pointers address (`aliases for') a given hunk of storage, it may happen that the storage is freed or reallocated (and thus moved) through
one alias and then referenced through another, which may lead to subtle (and possibly intermittent) lossage depending on the state and the
allocation history of the malloc \citeentry{arena}. Avoidable by use of allocation strategies that never alias allocated core, or by use of
higher-level languages, such as \citeentry{LISP}, which employ a garbage collector (see \citeentry{GC}). Also called a \citeentry{stale
pointer bug}. See also \citeentry{precedence lossage}, \citeentry{smash the stack}, \citeentry{fandango on core}, \citeentry{memory leak},
\citeentry{memory smash}, \citeentry{overrun screw}, \citeentry{spam}.

Historical note: Though this term is nowadays associated with C programming, it was already in use in a very similar sense in the Algol-60
and FORTRAN communities in the 1960s.

