\mainentry{avatar} n. Syn.

[in Hindu mythology, the incarnation of a god]
\begin{inparaenum}
	\item Among people working on virtual reality and \citeentry{cyberspace} interfaces, an \textbf{avatar} is an icon or representation of
		a user in a shared virtual reality. The term is sometimes used on \citeentry{MUD}s.
	\item {[}CMU, Tektronix] \citeentry{root}, \citeentry{superuser}. There are quite a few Unix machines on which the name of the superuser
		account is `avatar' rather than `root'. This quirk was originated by a CMU hacker who found the terms `root' and `superuser'
		unimaginative, and thought `avatar' might better impress people with the responsibility they were accepting.
\end{inparaenum}

