\mainentry{moby} /moh'bee/

[MIT: seems to have been in use among model railroad fans years ago. Derived
from Melville's ``Moby Dick'' (some say from `Moby Pickle').  Now common.]
\begin{inparaenum}
	\item adj. Large, immense, complex, impressive.
		\inlinequote{A Saturn V rocket is a truly moby frob.}
		\inlinequote{Some MIT undergrads pulled off a moby hack at the
			Harvard-Yale game.} (See \citeappendix{Appendix A} for discussion.)

	\item n. obs. The maximum address space of a machine (see below). For a
		680[234]0 or VAX or most modern 32-bit architectures, it is
		4,294,967,296 8-bit bytes (4 gigabytes).

	\item A title or address (never of third-person reference), usually used to
		show admiration, respect, and/or friendliness to a completent hacker.
		\inlinequote{Greetings, moby Dave. How's that address-book thing for
			the Mac going?}

	\item adj. In backgammon, doubles the dice, as in `moby sixes', `moby
		ones', etc. Compare this with \citesense{bignum}{3}: double sixes are
		both bignums and moby sixes, but moby ones are not bignums (the use of
		`moby' to describe double ones is sarcastic). Standard emphatic forms:
		`Moby foo', `moby win', `moby loss'. `Foby moo': a spoonerism due to
		Richard Greenblatt.

	\item The largest available unit of something which is available in
		discrete increments. Thus, ordering a ``moby Coke'' at the local
		fast-food joint is not just a request for a large Coke, it's an
		explicit request for the largest size they sell.
\end{inparaenum}

This term entered hackerdom with the Fabritek 256K memory added to the MIT AI
PDP-6 machine, which was considered unimaginably huge when it was installed in
the 1960s (at a time when a more typical memory size for a timesharing system
was 72 kilobytes). Thus, a moby is classically 256K 36-bit words, the size of a
PDP-6 or PDP-10 moby. Back when address registers were narrow the term was more
generally useful, because when a computer had virtual memory mapping, it might
actually have more physical memory attached to it than any one program could
access directly. One could then say \inlinequote{This computer has 6 mobies}
meaning that the ratio of physical memory to address space is 6, without having
to say specifically how much memory there actually is. That in turn implied
that the computer could timeshare six `full-sized' programs without having to
swap programs between memory and disk.  Nowadays the low cost of processor
logic means that address spaces are usually larger than the most physical
memory you can cram onto a machine, so most systems have much \textit{less}
space than one theoretical `native' moby of \citeentry{core}. Also, more modern
memory-management techniques (esp. paging) make the `moby count' less
significant. However, there is one series of widely-used chips for which the
term could stand to be revived -- the Intel 8088 and 80286 with their
incredibly \citeentry{brain-damaged} segmented-memory designs. On these, a
`moby' would be the 1-megabyte address span of a segment/offset pair (by
coincidence, a PDP-10 moby was exactly 1 megabyte of 9-bit-bytes).

