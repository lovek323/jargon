\mainentry{HCF} /H-C-F/ n.

Mnemonic for `Halt and Catch Fire', any of several undocumented and
semi-mythical machine instructions with destructive side-effects, supposedly
included for test purposes on several well-known architectures going as far
back as the IBM 360. The MC6800 microprocessor was the first for which an HFC
opcode became widely known. This instruction caused the processor to
\citeentry{toggle} a subset of the bus lines as rapidly as it could; in some
configurations this could actually cause lines to burn up. Compare
\citeentry{killer poke}.

\begin{new}
	\begin{usenet}
		Once upon a time there was a machine called the Compucolor II. It was
		made by Intecolor, a terminal manufacturer, and was deliberately
		crippled to keep it from competing with their terminals. They
		apparently were able to save about \$50 in hardware by giving the
		software more control over it, and cutting safety margins. The file
		system required al files be contiguous on disk: when you deleted a file
		it compressed the rest of the disk, using video RAM as a temporary
		buffer. Amusing, maybe. But not a very nice machine.

		OK, if you executed the following BASIC program, the display would
		vanish, it would make weird "singing" noises, and a burned smell would
		be immediately obvious. Turning off the power switch didn't have any
		effect: you have to pull the plug. Whether it would actually catch
		fire, I was too chicken to find out.

		The program?

		\begin{usenet}
			FOR I = 0 TO 255: OUT 6, I: NEXT I
		\end{usenet}

		Anyone know what this did, really?
		\citeusenet{da Silva, Peter}{peter@sugar.hackercorp.com}{1989}
			{Real life HCF code in this message}{alt.folklore.computers}
			{ZUsgBQklhJ0}{14 December}
	\end{usenet}
\end{new}

