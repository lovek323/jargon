\mainentry{ITS} /I-T-S/ n.

\begin{inparaenum}
	\item Incompatible Time-sharing System, an influential though highly idiosyncratic operating system written for PDP-6s and PDP-10s at
		MIT and long used at the MIT AI Lab. Much AI-hacker jargon derives from ITS folklore, and to have been `an ITS hacker' qualifies
		one instantly as an old-timer of the most venerable sort. ITS pioneered many important innovations, including transparent file
		sharing between machines and terminal-independent I/O. After about 1982, most actual work was shifted to newer machines, with the
		remaining ITS boxes run essentially as a hobby and service to the hacker community. The shutdown of the lab's last ITS machine in
		May 1990 marked the end of an era and sent old-timer hackers into mourning nationwide (see \citeentry{high moby})
	\item A mythical image of operating-system perfection worshiped by a bizarre, fervent retro-cult of old-time hackers and ex-users (see
		\citesense{troglodyte}{2}). ITS worshipers manage somehow to continue believing that an OS maintained by assembly-language
		hand-hacking that supported only monocase 6-character filenames in one directory per account remains superior to today's state of
		commercial art (their venom against \citeentry{Unix} is particularly intense). See also \citeentry{holy wars}, \citeentry{Weenix}.
\end{inparaenum}

