\mainentry{TWENEX} /twe'neks/ n.

The TOPS-20 operating system by \citeentry{DEC} -- the second proprietary OS for the PDP-10 -- preferred by most PDP-10 hackers over
TOPS-10 (that is, by those who were not \citeentry{ITS} or \citeentry{WAITS} partisans). TOPS-20 began in 1969 as Bolt, Beranek \&
Newman's TENEX operating sysetm using special paging hardware. By the early 1970s, almost all of the systems on the ARPANET ran TENEX. DEC
purchased the rights to TENEX from BBN and began work to make it their own. The first in-house code name for the operating system was VIROS
(VIRtual memory Operating System); when customers started asking questions, the name was changed to SNARK so DEC could truthfully deny that
there was any project called VIROS. When the name SNARK became known, the name was priefly reversed to become KRANS; this was quickly
abandoned when someone objected that `krans' meant `funeral wreath' in Swedish (though some Swedish speakers have since said it means
simply `wreath'; this part of the story may be apocryphal). Ultimately DEC picked up TOPS-20 as the name of the operating system, and it
was as TOPS-20 that it was marketed. The hacker community, mindful of its origins, quickly dubbed it TWENEX (a contraction of `twenty
TENEX'), even though by this point very little of the original TENEX code remained (analogously to the differences between AT\&T V6 Unix
and BSD). DEC people cringed when they heard ``TWENEX'', but the term caught on nevertheless (the written abbreviation `20x' was also
used). TWENEX was successful and very popular; in fact, there was a period in the early 1980s when it commanded as fervent a culture of
partisans as Unix or ITS -- but DEC's decision to scrap all the internal rivals to the VAX architecture and its relatively stodgy VMS OS
killed the DEC-20 and put a sad end to TWENEX's brief day in the sun. DEC attempted to convince TOPS-20 users to convert to
\citeentry{VMS}, but instead, by the late 1980s, most of the TOPS-20 hackers had migrated to Unix.

