\mainentry{Usenet} /yoos'net/ or /yooz'net/ n.

[from `Users Network'; the original spelling was USENET, but the mixed-case form is now widely preferred] A distributed \citeentry{bboard}
(bulletin board) system supported mainly by Unix hachines. Originally implemented in 1979-1980 by Steve Bellovin, Jim Ellis, Tom Truscott,
and Steve daniel at Duke University, it has swiftly grown to become international in scope and is now probably the largest decentralized
information utility in existence. As of early 1996, it hosts over 10,000 \citeentry{newsgroup}s and an average of over 500 megabytes (the
equivalent of several thousand paper pages) of new technical articles, news, discussion, chatter, and \citeentry{flamage} every day (and
that leaves out the graphics\dots).

By the year the Internet hit the mainstream (1994) the original UUCP transport for Usenet was fading out of use (see \citeentry{UUCPNET})
- almost all Usenet connections were over Internet links. A lot of newbies and journalists began to refer to ``Internet newsgroups'' as
though Usenet was and always had been just another Internet service. This ignorance greatly annoys experienced Usenetters. After the
advent of the World Wide Web, Web-based email clients (such as Gmail and Yahoo! mail), as well as online groups (Google Groups, though
they also contain Usenet archives), Usenet has fallen out of common usage, though some hackers still u se it. The signal-to-noise ratio
is still pretty low, many being spam posts from crackers and befuddled Google Groups users who think that Usenet was always just another
Google service.

