\mainentry{kluge} /klooj/

[from the German `klug', clever; poss. Polish `klucz' (a key, a hint, a mainpoint)]
\begin{inparaenum}
	\item n. A Rube Goldberg (or Heath Robinson) device, whether in hardware or software.
	\item n. A clever programming trick intended to solve a particularly nasty case in an expedient, if not clear, manner. Often used to repair bugs. Often involves \citeentry{ad-hockery} and verges on being a \citeentry{crock}.
	\item n. Something that works for the wrong reason.
	\item vt. To insert a kludge into a program. ``I've kluged this routine to get around that weird bug, but there's probably a better way.''
	\item {[}WPI]  n. A feature that is implemented in a \citeentry{rude} manner.
\end{inparaenum}

Nowadays this term is often encountered in the variant spelling `kludge'. Reports from \citeentry{old fart}s are consistent that `kluge' was the original spelling, reported around computers as far back as the mid-1950s and, at that time, exclusively of \textit{hardware} kluges. In 1947, the ``New York Folklore Quarterly'' reported a classic shaggy-dog story `Murgatroyd the Kluge Maker' then current in the Armed Forces, in which a `kluge' was a complex and puzzling artifact with a trivial function. Other sources report that `kluge' was common Navy slang in the WWII era for any piece of electronics that worked well on shore but consistently failed at sea.

However, there is a reason to believe this slang use may be a decade older. Several respondents have connected it to the brand name of a device called a ``Kluge paper feeder'', an adjunct to mechanical printing presses. Legend has it that the Kluge feeder was designed before small, cheap electric motors and control electronics; it relied on a fiendishly complex assortment of cams, belts, and linkages to both power and synchronize all its operations from one motive driveshaft. It was accordingly temperamental, subject to frequent breakdowns, and devilishly difficult to repair -- but oh, so clever! People who tell this story also aver that `Kludge' was the name of a design engineer.

There is in fact a Brandtjen \& Kluge Inc., an old family business that manufactures printing equipment - interestingly, their name is pronounced /kloo'gee/! Henry Brandtjen, president of the firm, told me (ESR, 1994) that his company was co-founded by his father and an engineer named Kluge /kloo'gee/, who built and co-designed the original Kluge automatic feeder in 1919. Mr. Brandtjen claims, however, that this was a \textit{simple} device (with only four cams); he says he has no idea how the myth of its complexity took hold.

\citeentry{TMRC} and the MIT hacker culture of the early '60s seems to have developed in a milieu that remembered and still used some WWII military slang (see also \citeentry{foobar}). It seems likely that `kluge' came to MIT via alumni of the many military electronics projects that had been located in Cambridge (many in MIT's venerable Building 20, in which \citeentry{TMRC} is also located) during the war.

The variant `kludge' was apparently popularized by the \citeentry{Datamation} article mentioned above; it was titled ``How to Design a Kludge'' (February 1962, pp. 30, 31). This spelling was probably imported from Great Britain, where \citeentry{kludge} has an independent history (though this fact was largely unknown to hackers on either side of the Atlantic before a mid-1993 debate in the Usenet group \textit{alt.folklore.computers} over the First and Second Edition versions of this entry; everybody used to think \citeentry{kludge} was just a mutation of \citeentry{kluge}). It now appears that the British, having forgotten the etymology of their own `kludge' when `kluge' crossed the Atlantic, repaid the U.S. by lobbing the `kludge' orthography in the other direction and confusing their American cousins' spelling!

The result of this history is a tangle. Many younger U.S. hackers pronounce the word as /klooj/ but spell it, incorrectly for its meaning and pronunciation, as `kludge'. (Phonetically, consider huge, centrifuge, and deluge as opposed to sludge, judge, budge, and fudge. Whatever its failings in other areas, English spelling is perfectly consistent about this distinction.) British hackers mostly learned /kluhj/ orally, use it in a restricted negative sense and are at least consistent. European hackers have mostly learned the word from written American sources and tend to pronounce it /kluhj/ but use the wider American spelling!

Some observers consider this mess appropriate in view of the word's meaning.

