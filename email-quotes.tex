\begin{center}\textbf{Email Quotes and Inclusion Conventions}\end{center}

One area where conventions for on-line writing are still in some flux is the marking of included material from earlier messages -- what
would be called `block quotations' in ordinary English. From the usual typographic convention employed for these (smaller font at an extra
indent), there derived a practice of included text being indented by one ASCII TAB (0001001) character, which under Unix and many other
environments gives the appearance of an 8-space indent.

Early mail and netnews readers had no facility for including messages this way, so people had to paste in copy manually. BSD Mail(1) was
the first message agent to support inclusion, and early Usenetters emulated its style. But the TAB character tended to push included text
too far to the right (especially in multiply nested inclusions), leading to ugly wraparounds. After a brief period of confusion (during
which an inclusion leading consisting of three or four spaces became established in EMACS and a few mailers), the use of leading $>$ or
$>$\ \ became standard, perhaps owing to uts use in ed(1) to display tabs (alternatively, it may derive from the $>$ that some early Unix
mailers used to quote lines starting with ``From'' in text, so they wouldn't look like the beginnings of new message headers). Inclusions
within inclusions keep their $>$ leaders, so the `nesting level' of a quotation is visually apparent.

The practice of including text from the parent article when opsting a followup helped solve what had been a major nuisance on Usenet: the
fact that articles do not arrive at different sites in the same order. Careless posters used to post articles that would give with, or even
consist entirely of, ``No, that's wrong'' or ``I agree'' or the like. It was hard to see who was responding to what. Consequently, around
1984, new news-posting software evolved a facility to automatically include the text of a previous article, marked with ``$>$'' -- but this
too has led to undesirable workarounds, such as the deliberate inclusion of zero-content filler lines which aren't quoted and thus pull the
message below the rejection threshold.

Because the default mailers supplied with Unix and other operating systems haven't evolved as quickly as human usage, the older conventions
using a leading TAB or three or four spaces are still alive; however, $>$-inclusion is now clearly the prevalent form in both netnews and
mail.

Inclusion practice is still evolving, and disputes over the `correct' inclusion style occasionally lead to \citeentry{holy wars}.

Most netters view an inclusion as a promise that comment on it will immediately follow. The preferred, conversational style looks like
this,

\begin{verbatim}
	> relevant exceprt 1
	response to excerpt
	> relevant excerpt 2
	response to excerpt
	> relevant excerpt 3
	response to excerpt
\end{verbatim}

or for short messages like this:

\begin{verbatim}
	> entire message
	response to message
\end{verbatim}

Thanks to poor design of some PC-based mail agents, one will occasionally see the entire quoted message after the response, like this

\begin{verbatim}
	response to message
	> entire message
\end{verbatim}

but this practice is strongly deprecated.

Though $>$ remains the standard inclusion leader, | is occasionally used for extended quotations where original variations in indentation
are being retained (one mailer even combines these and uses |$>$). One also sees different styles of quoting a number of authors in the
same message: one (deprecated because it loses information) uses a leader of $>$\ \ for everyone, another (the most commmon) is $>$ $>$
$>$ $>$ , $>$ $>$ $>$ , etc. (or $>$$>$$>$$>$ , $>$$>$$>$, etc., depending on the line length and nesting depth) reflecting the original
order of messages, and yet another is to use a different citation leader for each author, say $>$ , : , | , \}\ \ (preserving nesting so
that the inclusion order of messages is still apparent, or tagging the inclusions with authors' names). Yet another style is to use each
poster's initials (or login name) as a citatio leader for that poster.

Occasionally one sees a \# leader used for quotations from authoritative sources such as standards documents; the intended allusion is to
the root prompt (the special Unix command prompt issued when one is running as the privileged super-user).

