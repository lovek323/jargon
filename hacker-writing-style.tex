We've already seen that hackers often coin jargon by overgeneralizing grammatical rules. THis is one aspect of a more general fondness for
form-versus-content language jokes that shows up particularly in hackish writing. One correspondent reports that he consistently misspells
`wrong' as `worng'. Others have been known to criticize glitches in Jargon File drafts by observing (in the mode of Douglas Hofstadter)
``This sentence no verb'', or ``Too repetetetive'', or ``Bad speling'', or ``Incorrectspa cing.'' Similarly, intentional spoonerisms are
often made of phrases relating to confusion or things that are confusing: `drain bramage' for `brain damage' is perhaps the most common
(similarly, a hacker would be likely to write ``Excuse me, I'm cixelsyd today'', rather than ``I'm dyslexic today''). This sort of thing is
quite common and is enjoyed by all concerned.

Hackers tend to use quotes as balanced delimiters like parentheses, much to the dismay of American editors. Thus, if ``Jim is going'' is a
phrase, and so are ``Bill runs'' and ``Spock groks'', then hackers generally prefer to write: ``Jim is going'', ``Bill runs'' and ``Spock
groks''. This is incorrect according to standard American usage (which would put the commas and the final period inside the string quotes);
however, it is counter-intuitive to hackers to mutilate literal strings with characters that don't belong in them. Given the sorts of
examples that can come up in discussion of programming, American-style quoting can even be grossly misleading. When communicating command
lines or small pieces of code, extra characters can be a real pain in the neck.

Consider, for example, a sentence in a \citeentry{vi} tutorial that looks like this:

\begin{quote}
	Then delete a line from the file by typing ``dd''.
\end{quote}

Standard usage would make this

\begin{quote}
	Then delete a line from the file by typing ``dd.''
\end{quote}

but that would be very bad -- because the reader would be prone to type the string d-d-dot, and it happens that in vi(1) dot repeats the
last command accepted. The net result would be to delete two lines!

The Jargon File follows hackish usage throughout.

Interestingly, a similar style is now preferred practice in Great Britain, though the older style (which became stablished for
typographical reasons having to do with the aesthetics of comma and quotes in typeset text) is still accepted there.
\worktitle{Hart's Rules} and the \worktitle{Oxford Dictionary for Writers and Editors} call the hacker-like style `new' or `logical'
quoting. This turn British English to the style Latin languages (including Spanish, French, Italian, Catalan) have been using this style
all along.

Another hacker habit is a tendency to distinguish between `scare' quotes and `speech' quotes; that is, to use British-style single quotes
for marking and reserve American-style double quotes for actual reports of speech or text included from elsewhere. Interestingly, some
authorities describe this as correct general usage, but mainstream American English has gone to using double-quotes indiscriminately
enough that hacker usage appears marked [and, in fact, \textit{I} thought this was a personal quirk of mine until I checked with Usenet
--ESR]. One further permutation that is definitely not standard is a hackish tendency to do marking quotes by using apostrophes (single
quotes) in pairs; that is, 'like this'. This is modelled on string and character literal syntax in some programming languages (reinforced
by the fact that many character-only terminals display the apostrophe in typewriter style, as a vertical single quote).

One quirk that shows up frequently in the \citeentry{email} style of Unix hackers in particular is a tendency for some things that are
normally all-lowercase (including usernames and the names of commands and C routines) to remain uncapitalized even when they occur at the
beginning of sentences. It is clear that, for many hackers, the case of such identifiers becomes a part of their internal representation
(the `spelling') and cannot be overridden without mental effort (an appropriate reflex because Unix and C both distinguish cases and
confusing them can lead to \citeentry{lossage}). A way of escaping this dilemma is simply to avoid using these constructions at the
beginning of sentences.

There seems to be a meta-rule behind these nonstandard hackerisms to the effect that precision of expression is more important than
conformance to traditional rules; where the latter create ambiguity or lose information they can be discarded without a second though. It
is notable in this respect that other hackish inventions (for example, in vocabulary) also tend to carry very precise shades of meaning
even when constructed to appear slangy and loose. In fact, to a hacker, the contrast between `loose' form and `tight' content in jargon is
a substantial part of its humor!

Hackers have also developed a number of punctuation and emphasis conventions adapted to single-font all-ASCII communications links, and
these are occasionally carried over into written documents even when normal means of font changes, underlining, and the like are available.

One of these is that TEXT IN ALL CAPS IS INTERPRETED AS `LOUD', and this becomes such an ingrained synesthetic reflex that a person who
goes to caps-lock while in \citeentry{talk mode} may be aksed to ``stop shouting, please, you're hurting my ears!''.

Also, it is common to use bracketing with unusual characters to signify emphasis. The asterisk is most common, as in ``What the *hell*?''
even though this interferes with the common use of the asterisk suffix as a footnote mark. The underscore is also common, suggesting
underlining (this is particularly common with book titles; for example, ``It is often alleged that Joe Haldeman wrote \_The\_Forever\_War\_
as a rebuttal to Robert Heinlein's earlier novel of the future military, \_Starship\_Troopers\_.''). Other forms exemplified by ``=hell='',
``\textbackslash hell/'', or ``/hell/'' are occasionally seen (it's claimed that in the last example the first
slash pushes the letters over to the right to make them italic, and the second keeps them from falling over). On FidoNet, you might see
\#bright\# and \^{}dark\^{} text, which was actually interpreted by some reader software. Finally, words may also be emphasized L I K E T H
I S, or by a series of carets (\^{}) under them on the next line of the text.

There is a semantic difference between *emphasis like this* (which emphasizes the phrase as a whole), and *emphasis* *like* *this* (which
suggests the writer speaking very slowly and distinctly, as if to a very young child or a mentally impaired person). Bracketing a word with
the `*' character may also indicate that the writer wishes readers to consider that an action is taking place or that a sound is being
made. Examples: *bang*, *hic*, *rings*, *grin*, *kick*, *stomp*, *mumble*.

One might also see the above sound effects as $<$bang$>$, $<$hic$>$, $<$ring$>$, $<$grin$>$, $<$kick$>$, $<$stomp$>$, $<$mumble$>$. This
use of angle brackets to mark their contents originally derives from conventions used in \citeentry{BNF}, but since about 1993 it has been
reinforced by the HTML markup used on the World Wide Web.

Angle-bracket enclosure is also used to indicate that a term stands for some \citeentry{random} member of a larger class (this is straight
from \citeentry{BNF}). Examples like the following are common:

\begin{quote}
	So this $<$ethnic$>$ walks into a bar one day \dots
\end{quote}

There is also an accepted convention for `writing under erasure'; the text

\begin{quote}
	Be nice to this fool\^{}H\^{}H\^{}H\^{}Hgentleman,\\
	he's visiting from corporate HQ.
\end{quote}

reads roughly as ``Be nice to this fool, er, gentleman \dots''. This comes from the fact that the digraph \^{}H is often used as a print
representation for a backspace. It parallels (and may have been influenced by) the ironic use of `slashouts' in science-fiction fanzines.

A related habit uses editor commands to signify corrections to previous text. This custom is fading as more mailers get good editing
capabilities, but one occasionally still sees things like this:

\begin{quote}
	I've seen that term used on alt.foobar often.\\
	Send it to Erik for the File.\\
	Oops...s/Erik/Eric/.
\end{quote}

The s/Erik/Eric/ says ``change Erik to Eric in the preceding''. This syntax is borrowed from the Unix editing tools ed and sed, but is
widely recognized by non-Unix hackers as well.

In a formula, * signifies multiplication but two asterisks in a row are a shorthand for exponentiation (this derives from FORTRAN). Thus,
one might write 2 ** 8 = 256.

Another notation for exponentiation one sees more frequently uses the caret (\^{}, ASCII 1011110); one might write instead 2\^{}8 = 256.
This goes all the way back to Algol-60, which used the archaic ASCII `up-arrow' that later became the caret; this was picked up by Kemeny
and Kurtz's original BASIC, which in turn influenced the design of the bc(1) and dc(1) Unix tools, which have probably done most to
reinforce the convention on Usenet. (TeX math mode also uses \^{} for exponention.) The notation is mildly confusing to C programmers,
because \^{}\ means bitwise exclusive-or in C. Despite this, it was favored 3:1 over ** in a late-1990 snapshot of Usenet. It is used
consistently in this lexicon.

In on-line exchanges, hackers tend to use decimal forms or improper fractions (`3.5' or `7/2') rather than `typewriter style' mixed
fractions (`3-1/2'). The major motive here is probably that the former are more readable in a monospaced font, together with a desire to
avoid the risk that the latter might be read as `three minus one-half'. The decimal form is definitely preferred for fractions with a
terminating decimal representation; there may be some cultural influence here from the high status of scientific notation.

Another one-line convention, used especially for very large or very small numbers, is taken from C (which derived it from FORTRAN). This is
a form of `scientific notation' using `e' to replace `*10\^{}'; for example, one year is about 3e7 seconds long.

The tilde ($\Tilde$) is commonly used in a quantifying sense of `approximately'; that is, $\Tilde$50 means `about fifty'.

On Usenet and in the \citeentry{MUD} world, common C boolean, logical, and relational operators such as |, \&, ||, \&\&, ~, ==, ~+, $>$,
$<$, $>$= and =$<$ are often combined with English. The Pascal not-equals, $<$$>$, is also recognized, and occasionally one sees /= for
not-equals (from Ada, Common Lisp, and Fortran 90). The use of prefix `!' as a loose synonym for `not--' or `no--' is particularly common;
thus, `!clue' is read `no-clue' or `clueless'.

A related practice borrows syntax from preferred programming languages to express ideas in a natural-language text. For example, one might
see the following;

\begin{verbatim}
	In <jrh578689@thudpucker.com> J. R. Hacker wrote:
	>I recently had occasion to field-test the Snafu
	>Systems 2300E adaptive gonkulator. \ The price was
	>right, and the racing stripe on the case looked
	>kind of neat, but its performance left something
	>to be desired.

	Yeah, I tried one out too.
	
	#ifdef FLAME

	Hasn't anyone told those idiots that you can't get
	decent bogon suppression with AFJ filters at today's
	net volumes?
	#endif /* FLAME */

	I guess they figured the price premium for true
	frame-based semantic analysis was too high.
	Unfortunately, it's also the only workable approach.
	I wouldn't recommend purchase of this product unless
	you're on a *very* tight budget.

	#include <disclaimer.h>
	--
	                 == Frank Foonly (Fubarco Systems)
\end{verbatim}

In the above, the \#ifdef/\#endif pair is a conditional compilation syntax from C; here, it implies that the text between (which is a
\citeentry{flame}) should be evaluated only if you have turned on (or defiend on) the switch FLAME. The \#include at the end is C for
``include standard disclaimer here''; the `standard disclaimer' is understood to read, roughly, ``These are my personal opinions and not to
be construed as the official position of my employer.''

\begin{new}
	\begin{usenet}
		Disclaimer: \_I\_ don't know what I said, much less my employer
		\citeusenet{Klop, Kevin}{unknown}{1989}{this really happened . . . (long)}{alt.folklore.computers}{oPCpIOR\_cl0}{15 December}
	\end{usenet}

	\begin{usenet}
		I'm now employed, but I'm responsible for my employer's opinions, not vice versa.
		\citeusenet{Willey, James P.}{willey@arrakis.NEVADA.EDU}{1989}{Re: HCF instruction (was Re: Welcome}{alt.folklore.computers}
			{-2dkTWD8\_PI}{17 December}
	\end{usenet}

	Some users poke fun at this convention:

	\begin{usenet}
		Exclaimer:\ \ Hey!
		\citeusenet{Casseres, David}{unknown}{1989}{You're Fired!}{alt.folklore.computers}{7axbqbNfDkM}{15 December}
	\end{usenet}
\end{new}

The top section in the example, with $>$ at the left margin, is an example of an inclusion convention we'll discuss below.

More recently, following on the huge popularity of the World Wide Web, psueudo-HTML markup has become popular for similar purposes:

\begin{verbatim}
	<flame>
	Your father was a hamster and your mother smelt of elderberries!
	</flame>
\end{verbatim}

You'll even see this with an HTML-style modifier:

\begin{verbatim}
	<flame intensity="100%">
	You seem well-suited for a career in government.
	</flame>
\end{verbatim}

Another recent (late 1990s) construction now common on USENET seems to be borrowed from Perl. It consists of using a dollar sign before an
uppercase form of a word or acronym to suggest any \citeentry{random} member of the class indicated by the word. Thus: `\$PHB' means ``any
random member of the class `Pointy-Haired Boss' ''.

Hackers also mix letters and numbers more freely than in mainstream usage. In particular, it is good hackish style to write a digit
sequence when you intend the reader to understand the text string that names that number in English. So, hackers prefer to write `1970s'
rather than `nineteen-seventies' or `1970's' (the latter looks like a possessive).

It should also be noted that hackers exhibit much less reluctance to use multiply nested parentheses than is normal in English. Part of
this is almost certainly due to influence from LISP (which uses deeply nested parentheses (like this (see?)) in its syntax a lot), but it
has also been suggested that a more basic hacker trait of enjoying playing with complexity and pushing systems to their limits is in
operation.

Finally, it is worth mentioning that many studies of on-line communication have shown that electronic links have a de-inhibiting effect on
people. Deprived of the body-language cues through which emotioanl state is expressed, people tend to forget everything about other parties
except what is presented over that ASCII link. This has both good and bad effects. A good one is that it encourages honesty and tends to
break down hierarchical authority relationships; a bad one is that it may encourage depersonalization and gratuitous rudeness. Perhaps in
response to this, experienced netters often display a sort of conscious formal politesse in their writing that has passed out of fashion in
other spoken and written media (for example, the phrase ``Well said, sir!'' is not uncommon).

Many introverted hackers who are next to inarticulate in person communicate with considerable fluency over the net, perhaps precisely
because they can forget on an unconscious level that they are dealing with people and thus don't feel stressed and anxious as they would
face to face.

Though it is considered gauche to publicly criticize posters for poor spelling or grammar, the network places a premium on literacy and
clarity of expression. It may well be that future historians of literature will see in it a revival of the great tradition of personal
letters as art.

