Entries are sorted in case-blind ASCII collation order (rather than the letter-by-letter order ignoring interword spacing common in
mainstream dictionaries), except that all entries beginning with nonalphabetic characters are sorted after Z. The case-blindness is a
feature, not a bug.

The beginning of each entry is marked by a colon (:) at the left margin. This convention helps out tools like hypertext browsers that
benefit from knowing where entry boundaries are, but aren't as context-sensitive as humans.

In pure ASCII renderings of the Jargon File, you will see \{\} used to bracket words which themselves have entries in the File. This isn't
done all the time for every such word, but it is done everywhere that a reminder seems useful that the term has a jargon meaning and one
might wish to refer to its entry.

In this all-ASCII version, headwords for topic entries are distinguished from those for ordinary entries by being followed by ``::'' rather
than ``:''; similarly, references are surrounded by ``\{\{'' and ``\}\}''.

Defining instances of terms and phrases appear in `slanted type'. A defining instance is one which occurs near to or as part of an
explanation of it.

Prefixed ** is used as linguists do; to mark examples of incorrect usage.

We follow the `logical' quoting convention described in the Writing Style section above. In addition, we reserve double quotes for actual
excerpts of text or (sometimes invented) speech. Scare quoets (which mark a word being used in a nonstandard way), and philosopher's quotes
(which turn an utterance into the string of letters or words that name it) are both rendered with single quotes.

References such as malloc(3) and patch(1) are to Unix facilities (some of which, such as patch(1) are actually freeware distributed over
Usenet). The Unix manuals use foo(n) to refer to item foo in section (n) of the manual, where n=1 is utilities, n=2 is system calls, n=3 is
C library routines, n=6 is games, and n=8 (where present) is system administration utilities. Sections 4, 5, and 7 of the manuals have
changed roles frequently and in any case are not referred to in any of the entries.

Various abbreviations used frequently in the lexicon are summarized here:

\begin{multicols}{3}
	\textbf{abbrev.} abbreviation

	\textbf{adj.} adjective

	\textbf{adv.} adverb

	\textbf{alt.} alternate

	\textbf{cav.} caveat

	\textbf{conj.} conjunction

	\textbf{esp.} especially

	\textbf{excl.} exclamation

	\textbf{imp.} imperative

	\textbf{interj.} interjection

	\textbf{n.} noun

	\textbf{obs.} obsolete

	\textbf{pl.} plural

	\textbf{pref.} prefix

	\textbf{prob.} probably

	\textbf{prov.} proverbial

	\textbf{quant.} quantifier

	\textbf{suff.} suffix

	\textbf{syn.} synonym (or synonymous with)

	\textbf{v.} verb (may be transitive or intransitive)

	\textbf{var.} variant

	\textbf{vi.} intransitive verb

	\textbf{vt.} transitive verb
\end{multicols}

Where alternate spellings or pronunciations are given, alt. separates two possibilities with nearly equal distribution, while var. prefixes
one that is markedly less common than the primary.

Where a term can be attributed to a particular subculture or is known to have originated there, we have tried to so indicate. Here is a
list of abbreviations used in etymologies:

\begin{multicols}{2}
	\paragraph{Amateur Packet Radio} A technical culture of ham-radio sites using AX.25 and TCP/IP for wide-area networking and BBS systems.

	\paragraph{Berkeley} University of California at Berkeley

	\paragraph{BBN} Bolt, Beranek \& Newman

	\paragraph{Cambridge} the university in England (not the city in Massachusetts where MIT happens to be located!)

	\paragraph{CMU} Carnegie-Mellon University

	\paragraph{Commodore} Commodore Business Machines

	\paragraph{DEC} The Digital Equipment Corporation (now Compaq).

	\paragraph{Fairchild} The Fairchild Instruments Palo Alto development group

	\paragraph{FidoNet} See the \citeentry{FidoNet} entry

	\paragraph{IBM} International Business Machines

	\paragraph{MIT} Massachusetts Institute of Technology; esp. the legendary MIT AI Lab culture of roughly 1971 to 1983 and its feeder
		groups, including the Tech Model Railroad Club
	
	\paragraph{NRL} Naval Research Laboratories

	\paragraph{NYU} New York University

	\paragraph{OED} The Oxford English Dictionary

	\paragraph{Purdue} Purdue University

	\paragraph{SAIL} Stanford Artificial Intelligence Laboratory (at Stanford University)

	\paragraph{SI} From Syst\'{e}me International, the name for the standard conventions of metric nomenclature used in the sciences

	\paragraph{Stanford} Stanford University

	\paragraph{Sun} Sun Microsystems

	\paragraph{TMRC} Some MITisms go back as far as the Tech Model Railroad Club (TMRC) at MIT c. 1960. Material marked TMRC is from ``An
		Abridged Dictionary of the TMRC Language'', originally compiled by Pete Samson in 1959

	\paragraph{UCLA} University of California at Los Angeles

	\paragraph{UK} the United Kingtom (England, Wales, Scotland, Northern Ireland)

	\paragraph{Usenet} See the \citeentry{Usenet} entry

	\paragraph{WPI} Worcester Polytechnic Institute, site of a very active community of PDP-10 hackers during the 1970s

	\paragraph{WWW} The World-Wide-Web.

	\paragraph{XEROC PARC} XEROX's Palo Alto Research Center, site of much pioneering research in user interface design and networking

	\paragraph{Yale} Yale University
\end{multicols}

Some other etymology abbreviatons such as \citeentry{Unix} and \citeentry{PDP-10} refer  to technical cultures surrounding specific operating sytems,
processors, or other environments. The fact that a term is labelled with any one of these abbreviations does not necessarily mean its use
is confined to that culture. In particular, many terms labelled `MIT' and `Stanford' are in quite general use. We have tried to give some
indication of the distribution of speakers in the usage notes; however, a number of factors mentioned in the introduction conspire to make
these indications less definite than might be desirable.

A few new definitions attached to entries are marked [proposed]. These are usually generalizations suggested by editors or Usenet
respondents in the process of commenting on previous definitinos of those entries. These are not represented as established jargon.

