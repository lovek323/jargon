Pronunciation keys are provided in the jargon listings for all entries that are neither dictionary words pronounced as in standard
English nor obvious compounds thereof. Slashes bracket phonetic pronunciations, which are to be interpreted using the following
conventions:

\renewcommand{\labelenumi}{\arabic{enumi}.}
\begin{enumerate}
	\item Syllables are hyphen-separated, except that an accent or back-accent follows each accented syllable (the back-accent marks a
		secondary accent in some words of four or more syllables). If no accent is given, the word is pronounced with equal accentuation on
		all syllables (this is common for abbreviations).
	\item Consonants are pronounced as in American English. The letter `g' is always hard (as in ``got'' rather than ``giant''); `ch' is
		soft (``church'' rather than ``chemist''). The letter `j' is the sound that occurs twice in ``judge''. The letter `s' is always as
		in ``pass'', never a z sound. The digraph `kh' is the guttural of ``loch'' or ``l'chaim''. The digraph `gh' is the aspirated g+h of
		``bughouse'' or ``ragheap'' (rare in English).
	\item Uppercase letters are pronounced as their English letter names; thus (for example) /H-L-L/ is equivalent to /aych el el/. /Z/ may
		be pronounced /zee/ or /zed/ depending on your local dialect.
	\item Vowels are represented as follows:
		\begin{quote}
			back, that\\
			father, palm (see note)\\
			far, mark\\
			flaw, caught\\
			bake, rain\\
			less, men\\
			easy, ski\\
			their, software\\
			trip, hit\\
			life, sky\\
			block, stock (see note)\\
			flow, sew\\
			loot, through\\
			more, door\\
			out, how\\
			boy, coin\\
			but, some\\
			put, foot\\
			yet, young\\
			few, chew\\
			/oo/ with optional fronting as in `news' (/nooz/ or /nyooz/)
		\end{quote}
\end{enumerate}
\renewcommand{\labelenumi}{\colorbox{shadecolor}{{\sffamily\footnotesize\arabic{enumi}\label{\entrylabel-\theenumi}}}}

The glyph /*/ is used for the `schwa' sound of unstressed or occluded vowels (the one that is often written with an upside-down `e'). The
schwa vowel is omitted in syllables containing vocalic r, l, m or n; that is, in `kitten' and `color' would be rendered /kit'n/ and
/kuhl'r/, not /kit'*n/ and /kuhl'*r/.

Note that the above table reflects mainly distinctions found in standard American English (that is, the neutral dialect spoken by TV
network announcers and typical of educated speech in the Upper Midwest, Chicago, Minneapolis/St. Paul and Philadelphia). However, we
separate /o/ from /ah/, which tend to merge in standard American. This may help readers accustomed to accents resembling British Received
Pronunciation.

The intent of this sceme is to permit as many readers as possible to map the pronunciations into their local dialect by ignoring some
subset of the distinctions we make. Speakers of British RP, for example, can smash terminal /r/ and all unstressed vowels. Speakers of many
varieties of southern American will automatically map /o/ to /aw/; and so forth. (Standard American makes a good reference dialect for this
purpose because it has crisp consonants and more vowel distinctions than other major dialects, and tends to retain distinctions between
unstressed vowels. It also happens to be what your editor speaks.)

Entries with a pronunciation of '//' are written-only usages. (No, Unix weenies, this does not mean `pronounce like previous
pronunciaton'!)

