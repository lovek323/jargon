Although the Jargon File remains primarily a lexicon of hacker usage in American English, we have made some effort to get input from
abroad. Though the hacker-speak of other languages often uses translations of jargom from English (often as transmitted to them by earlier
Jargon File versions!), the local variations are interesting, and knowledge of them may be of some use to travelling hackers.

There are some references herein to `Commonwealth hackish'. These are intended to describe some variations in hacker usage reported in
the English spoken in Great Britain and the Commonwealth (Canada, Australia, India, etc. -- though Canada is heavily influenced by
American usage). There is also an entry on \citeentry{Commonwealth Hackish} reporting some general phonetic and vocabulary differences from
U.S. hackish.

Hackers in Western Europe and (especially) Scandinavia report that they often use a mixture of English and their native languages for
technical conversation. Occasionally they develop idioms in their English usage that are influenced by their native-language styles. Some
of these are reported here.

On the other hand, English often gives rise to grammatical and vocabulary mutations in the native language. For example, Italian hackers
often use the nonexistent verbs `scrollare' (to scroll) and `deletare' (to delete) rather than native Italian `scorrere' and `cancellare'.
Similarly, the English verb `to hack' has been conjugated in Swedish. And Spanish-speaking hackers use `linkar' (to link), `debuggar' (to
debug), and `lockear' (to lock).

European hackers report that this happens parly because the English terms make finger distinctions than are available in their native
vocabularies, and partly because deliberate language-crossing makes for amusing wordplay.

A few notes on hackish usage in Russian have been added where they are parallel with English idioms and thus comprehensible to
English-speakers.

