\section*{Some AI Koans}\label{Some-AI-Koans}

These are some of the funnies examples of a genre of jokes told at the MIT AI Lab about various noted hackers. The original koans were
composed by Danny Hillis, who would later found Connection Machines, Inc. In reading these, it is at least useful to know that Minsky,
Sussman, and Drescher are AI researchers of note, that Tom Knight was one of the Lisp machine's principal designers, and that David Moon
wrote much of Lisp Machine Lisp.

\begin{center}* * *\end{center}

A novice was trying to fix a broken Lisp machine by turning the power off and on.

Knight, seeing what the student was doing, spoke sternly: ``You cannot fix a machine by just power-cycling it with no understanding of what
is goign wrong.''

Knight turned the machine off and on.

The machine worked.

\begin{center}* * *\end{center}

One day a student came to Moon and said: ``I understand how to make a better garbage collector. We must keep a reference count of the
pointers to each cons.''

Moon patiently told the student the following story:

\hspace{3em}``One day a student came to Moon and said: `I understand how to make a better garbage collector\dots' ''

[Ed. note: Pure reference-count garbage collectors have problems with circular structures that point to themselves.]

\begin{center}* * *\end{center}

In the days when Sussman was a novice, Minsky once came to him as he sat hacking at the PDP-6.

``What are you doing?'', asked Minsky.

``I am training a randomly wired neural net to play Tic-Tac-Toe'' Sussman replied.

``Why is the net wired randomly?'', asked Minstky.

``I do not want it to have any preconceptions of how to play'', Sussman said.

Minsky then shut his eyes.

``Why do you close your eyes?'', Sussman asked his teacher.

``So that the room will be empty.''

At that moment, Sussman was enlightened.

\begin{center}* * *\end{center}

A disciple of another sect once came to Drescher as he was eating his morning meal.

``I would like to give you this personality test'', said the outsider, ``because I want you to be happy.''

Drescher took the paper that was offered him and put it in the toaster, saying: ``I wish the toaster to be happy, too.''

