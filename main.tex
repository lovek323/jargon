\documentclass{report}

\usepackage{url,paralist,fancyhdr,xcolor,enumitem,textcomp,verbatim,longtable,multicol,scrextend,framed}
\usepackage{xstring/xstring}
\usepackage[margin=1in]{geometry}
\usepackage[titles]{tocloft}
\usepackage[T1]{fontenc}

\setlength{\cftbeforechapskip}{6pt}

\newcommand*{\mainentry}[1]{%
    \paragraph{#1}%
    \markboth{#1}{#1}%
    \StrSubstitute{#1}{ }{-}[\entrylabel]%
    \StrSubstitute{\entrylabel}{`}{}[\entrylabel]%
    \StrSubstitute{\entrylabel}{'}{}[\entrylabel]%
    \StrSubstitute{\entrylabel}{---}{-}[\entrylabel]%
    \label{\entrylabel}}
\newcommand*{\citeentry}[1]{%
    \StrSubstitute{#1}{ }{-}[\temp]%
    \StrSubstitute{\temp}{`}{}[\temp]%
    \StrSubstitute{\temp}{'}{}[\temp]%
    \StrSubstitute{\temp}{---}{-}[\temp]%
    {\small\sffamily\bfseries#1}\superscript{\rmfamily\mdseries[\pageref{\temp}]}}
\newcommand*{\citesense}[2]{%
    \StrSubstitute{#1-#2}{ }{-}[\temp]%
    {\small\sffamily\bfseries#1 (sense #2)}\superscript{\rmfamily\mdseries[\pageref{\temp}]}}
\newcommand*{\Chapter}[2]{%
    \StrSubstitute{#1}{ }{-}[\temp]%
    \chapter*{#1}\label{chapter-\temp}
    \addcontentsline{toc}{chapter}{#1\protect{\rmfamily\mdseries\\ #2}}}
\newcommand{\inlinequote}[1]{%
	{$\cdot$\ \small``#1''}}
\newcommand{\superscript}[1]{{\scriptsize#1}}
\newcommand{\worktitle}[1]{\textit{#1}}
\newcommand{\Tilde}{{\raise.17ex\hbox{$\scriptstyle\sim$}}}

%\citeusenet{Name, Poster}{poster@email}{YYYY}{Post subject}{group.name}{googleid}{D Month}
\newcommand{\citeusenet}[7]
{\hfill{\rmfamily\footnotesize[\mbox{#1} (#3): #4, {\bfseries#5}, \mbox{#7 #3}]}}
\newcommand{\citeusenetfn}[7]{{\footnotesize\rmfamily{(#1, #3: #4 \textit{#5}, #7})}}
\newenvironment{usenet}{\begin{addmargin}{0.5in}\raggedright\small\ttfamily}{\rmfamily\end{addmargin}}

\definecolor{shadecolor}{HTML}{DDDDDD}
\newenvironment{new}{\begin{oframed}{\footnotesize NOT IN THE JARGON FILE

}}{\end{oframed}}

\newenvironment{code}{\begin{addmargin}{0.5in}\raggedright\ttfamily}{\rmfamily\end{addmargin}}

\renewcommand{\labelenumi}{\colorbox{shadecolor}{{\sffamily\footnotesize\arabic{enumi}\label{\entrylabel-\theenumi}}}}

\begin{document}
    \title{The New Hacker's Dictionary}
    \author{Edited by Yash Tulsyan}

    \setlength{\parskip}{0pt}
    \setlength{\parindent}{0pt}

    \input preamble

    \tableofcontents

    \Chapter{Introduction}{The purpose and scope of this File}
	\input introduction
    \Chapter{A Few Terms}{Of slang, Jargon and Techspeak}
	\input a-few-terms
    \Chapter{Revision History}{How the File came to be}
	\input revision-history

    \part{How Jargon Works}
    \Chapter{Jargon Construction}{How hackers invent jargon}
	\input jargon-construction
    \Chapter{Hacker Writing Style}{How they write}
	\input hacker-writing-style
    \Chapter{Email Quotes}{And the Inclusion Problem}
	\input email-quotes
    \Chapter{Hacker Speech Style}{How hackers talk}
	\input hacker-speech-style
    \Chapter{International Style}{Some notes on usage outside the U.S.}
	\input international-style
    \Chapter{Lamer-speak}{Crackers, Phreaks, and Lamers}
	\input lamer-speak

    \part{How to Use the Lexicon}
    \Chapter{Pronunciation Guide}{How to read the pronunciation keys}
	\input pronunciation-guide
    \Chapter{Other Lexicon Conventions}{How to read lexicon entries}
	\input other-lexicon-conventions
    \Chapter{Format for New Entries}{How to submit new entries for the File}
	\input format-for-new-entries

    \addtocontents{toc}{\protect\vspace{6pt}}
    \Chapter{The Jargon Lexicon}{The lexicon itself}

    \pagestyle{fancy}
    \fancyhead{}
    \fancyfoot{}
    \fancyhead[L,R]{%
        \sffamily\bfseries\rightmark--\leftmark}
    \fancyhead[R]{\sffamily\thepage}
    \fancyhead[L]{\sffamily\thepage}
    \renewcommand{\headrulewidth}{0.1em}

    \mainentry{ad-hockery}

	\Chapter{A}{}
	\Chapter{B}{}
	\input entries/b
	\Chapter{C}{}
	\input entries/c
	\Chapter{D}{}
	\input entries/d
	\Chapter{E}{}
	\input entries/e
	\Chapter{F}{}
	\input entries/f
	\Chapter{G}{}
	\input entries/g
	\Chapter{H}{}
	\input entries/h
	\Chapter{I}{}
	\input entries/i
	\Chapter{J}{}
	\Chapter{K}{}
	\input entries/k
	\Chapter{L}{}
	\input entries/l
	\Chapter{M}{}
    \input entries/m
	\Chapter{N}{}
    \input entries/n
	\Chapter{O}{}
    \input entries/p
	\Chapter{P}{}
	\input entries/r
	\Chapter{Q}{}
	\Chapter{R}{}
	\Chapter{S}{}
	\Chapter{T}{}
	\input entries/t
	\Chapter{U}{}
	\input entries/u
	\Chapter{V}{}
	\input entries/v
	\Chapter{W}{}
	\input entries/w
	\Chapter{X}{}
	\Chapter{Y}{}
	\Chapter{Z}{}

    \mainentry{old fart}

    \mainentry{Stupids}

    \mainentry{suit} n.

    \begin{inparaenum}
        \item Ugly and uncomfortable `business clothing' often worn by non-hackers. Invariably worn with a `tie', a strangulation device that partially cuts off the blood supply to the brain. It is thought that this explains much about the behavior of suit-wearers. Compare \citeentry{droid}.
        \item A person who habitually wears suits, as distinct from a techie or hacker. See \citeentry{pointy-haired}, \citeentry{burble}, \citeentry{management}, \citeentry{Stupids}, \citeentry{SNAFU principle}, \citeentry{PHB}, and \citeentry{brain-damaged}.
    \end{inparaenum}

    \mainentry{SNAFU principle}


    \pagestyle{plain}

    \appendix

    \chapter*{Appendix A}\label{Appendix-A}
	\input appendix-a

    \chapter*{Appendix B}\label{Appendix-B}

    \chapter*{Appendix C}\label{Appendix-C}

    \chapter*{Bibliography}\label{Bibliography}
\end{document}

