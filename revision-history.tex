The original Jargon File was a collection of hacker jargon from technical
cultures including the MIT AI Lab, the Stanford AI lab (SAIL), and others of
the old ARPANET AI/LISP/PDP-10 communities including Bolt, Beranek, and Newman
(BBN), Carnegie-Mellon University (CMU), and Worcester Polytechnic Institute
(WPI).

The Jargon File (hereafter referred to as `jargon-1' or `the File') was begun
by Raphael Finkel at Stanford in 1975. From this time until the plug was
finally poulled on the SAIL computer in 1991, the File was named
AIWORD.RF[UP,DOC] there. Some terms in it date back considerably earlier
(\citeentry{frob} and some senses of \citeentry{moby}, for instance, go back to
the Tech Model Railroad Club at MIT and are believed to date at least back to
the early 1960s). The revisions of jargon-1 were all unnumbered and may be
collectively considered `Version 1'.

In 1976, Mark Crispin, having seen an announcement about the File on the SAIL
computer, \citeentry{FTP}ed a copy of the File to MIT. He noticed that it was
hardly restricted to `AI words' and so stored the file on his directory as
AI:MRC;SAIL JARGON.

The file was quickly renamed the JARGON $>$ (the `$>$' caused versioning under
ITS) as a flurry of enhancements were made by Mark Crispin and Guy L. Steele
Jr. Unfortunately, amidst all this activity, nobody thought of correcting the
term `jargon' to `slang' until the compendium had already become widely known
as the Jargon File.

Raphael Finkel dropped out of active participation shortly thereafter and Don
Woods became the SAIL contact for the File (which was subsequently kept in
duplicate at SAIL and MIT, with periodic resynchronizations).

The File expanded by fits and starts until about 1983; Richard Stallman was
prominent among the contributors, adding many MIT and ITS-related coinages.

In Spring 1981, a hacker named Charles Spurgeon got a large chunk of the File
published in Stewart Brand's \worktitle{CoEvolution Quarterly} (issue 29, pages
26--35) with illustrations by Phil Wadler and Guy Steele (including a couple of
the Crunchly cartoons). This appears to have been the File's first paper
publication.

A late version of jargon-1, expanded with commentary for the mass market, was
edited by Guy Steele into a book published in 1983 as \worktitle{The Hacker's
Dictionary} (Harper \& Row CN 1082, ISBN 0-06-091082-8). The other jargon-1
editors (Raphael Finkel, Don Woods, and Mark Crispin) contributed to this
revision, as did Richard M. Stallman and Geoff Goodfellow. This book (now out
of print) is hereafter referred to as `Steele-1983' and those six as the
Steele-1983 coauthors.

Shortly after the publication of Steele-1983, the File effectively stopped
growing and changing. Originally, this was due to a desire to freeze the file
temporarily to facilitate the production of Steele-1983, but external
conditions caused the `temporary' freeze to become permanent.

The AI Lab culture had been hit hard in the late 1970s by funding cuts and the
resulting administrative decision to use vendor-supported hardware and software
instead of homebrew whenever possible. At MIT, most AI work had turned to
dedicated LISP Machines. At the same time, the commercialization of AI
technology lured some of the AI Lab's best and brightest away to startups along
the Route 128 strip in Massachusetts and out West in Silicon Valley. The
startups built LISP machines for MIT; the central MIT-AI computer became a
\citeentry{TWENEX} system rather than a host for the AI hackers' beloved
\citeentry{ITS}.

The Stanford AI Lab had effectively ceased to exist by 1980, although the SAIL
computer continued as a Computer Science Department resource until 1991.
Stanford became a major \citeentry{TWENEX} site, at one point operating more
than a dozen TOPS-20 systems; but by the mid-1980s most of the interesting
software work was being done on the emerging BSD Unix standard.

In April 1983, the PDP-10-centered cultures that had nourished the File were
dealt a death-blow by the cancellation of the Jupiter project at Digital
Equipment Corpiration. The  File's compilers, already dispersed, moved on to
other things. Steele-1983 was partly a monument to what its authors thought was
a dying tradition; no one involved realized at the time just how wide its
influence was to be.

By the mid-1980s the File's content was dated, but the legend that had grown up
around it never quite died out. The book, and softcopies obtained off the
ARPANET, circulated even in cultures far removed from MIT and Stanford; the
content exerted a strong and continuing influence on hacker language and humor.
Even as the advent of the microcoputer and other trends fueled a tremendous
expansion of hackerdom, the File (and related materials such as the
\citeappendix{Some AI Koans} in \citeappendix{Appendix A}) came to be seen as a
sort of sacred epic, a hacker-culture Matter of Britain chronicling the heroic
exploits of the Knights of the Lab. The pace of change in hackrdom at large
accelerated tremendously -- but the Jargon File, having passed from living
document to icon, remained essentially untouched for seven years.

Eric S. Raymond \url{<esr@snark.thyrsus.com>} maintained the File with
assistance from Guy L. Steele Jr. \url{<gls@think.com>} between 1991 and 2011.
However, as of 2003, this file was contaminated by Eric Raymond's politics (as
noted by frustrated hackers, including Richard Stallman). Complains were many
(some can be found here \url{<http://www.ntk.net/2003/06/06/>}) and, until 2011
there were no revisions, major or minor. The various revisions are referred to
here by their file numbers, but the Raymond period in general will hereafter be
referred to as \worktitle{ESR-2003}.

ESR-2003 contained nearly the entire text of a late version of jargon-1 (a few
obsolete PDP-10-related entries were dropped after careful consultation with
the editors of Steele-1983). It merged in about 80\% of the Steele-1983 text,
omitting some framing material and very few entries introduced in Steele-1983
that are now obsolete.

ESR-2003 casted a wider net than the old Jargon File; its aim is to cover not
just AI or PDP-10 hacker culture but all the technical computing cultures
wherein the true hacker-nature is manifested. More than half of the entries now
derive from \citeentry{Usenet} and represent jargon now current in the C and
Unix communities, but special efforts have been made to collect jargon from
other cultures including IBM PC programmers, Amiga fans, Mac enthusiasts, and
even the IBM mainframe world.

The 2.9.6 version became the main text of \worktitle{The New Hacker's
Dictionary}, by Eric Raymond (ed.), MIT Press 1991, ISBN 0-262-68069-6.

The 3.0.0 version was published in September 1993 as the second edition of
\worktitle{The New Hacker's Dictionary}, again from MIT Press (ISBN
0-262-18154-1).

If you want the book, you should be able to find it at any of the major
bookstore chains. Failing that, you can order by mail from The MIT Press 55
Hayward Street Cambridge, MA 02142 or order by phone at (800)-356-0343 or
(617)-625-8481.

The Jargon File is currently maintained by Yash Tulsyan
\url{<yashtulsyan@gmail.com>} in response to the static nature of ESR-2003, as
well as the many criticisms layed out against it. It was forked from version
4.2.0.

The maintainer(s) are committed to updating the on-line version of the Jargon
File through and beyond paper publication, and will continue to make it
available to archives and public-access sites as a trust of the hacker
community.

Here is a chronology of the high points in the on-line revisions since
Steele-1983:

Version 2.1.1, Jun 12 1990: the Jargon File comes alive again after a
seven-year hiatus. Reorganization and massive additions were by Eric S.
Raymond, approved by Guy Steele. Many items of UNIX, C, USENET, and
microcomputer-based jargon were added at that time.

Version 2.9.6, Aug 16 1991: corresponds to reproduction copy for book. This
version had 18952 lines, 148629 words, 975551 characters, and 1702 entries.

Version 2.9.7, Oct 28 1991: first markup for hypertext browser. This version
had 19432 lines, 152132 words, 999595 characters, and 1750 entries.

Version 2.9.8, Jan 01 1992: first public release since the book, including over
fifty new entries and numerous corrections/additions to old ones. Packaged with
version 1.1 of vh(1) hypertext reader. This version had 19509 lines, 153108
words, 1006023 characters, and 1760
entries.

Version 2.9.9, Apr 01 1992: folded in XEROX PARC lexicon. This version had
20298 lines, 159651 words, 1048909 characters, and 1821 entries.  Version
2.9.10, Jul 01 1992: lots of new historical material. This version had 21349
lines, 168330 words, 1106991 characters, and 1891 entries.

Version 2.9.11, Jan 01 1993: lots of new historical material. This version had
21725 lines, 171169 words, 1125880 characters, and 1922 entries.

Version 2.9.12, May 10 1993: a few new entries \& changes, marginal MUD/IRC
slang and some borderline techspeak removed, all in preparation for 2nd Edition
of TNHD. This version had 22238 lines, 175114 words, 1152467 characters, and
1946 entries.

Version 3.0.0, Jul 27 1993: manuscript freeze for 2nd edition of TNHD. This
version had 22548 lines, 177520 words, 1169372 characters, and 1961 entries.

Version 3.1.0, Oct 15 1994: interim release to test WWW conversion. This
version had 23197 lines, 181001 words, 1193818 characters, and 1990 entries.

Version 3.2.0, Mar 15 1995: Spring 1995 update. This version had 23822 lines,
185961 words, 1226358 characters, and 2031 entries.

Version 3.3.0, Jan 20 1996: Winter 1996 update. This version had 24055 lines,
187957 words, 1239604 characters, and 2045 entries.

Version 3.3.1, Jan 25 1996: Copy-corrected improvement on 3.3.0 shipped to MIT
Press as a step towards TNHD III. This version had 24147 lines, 188728 words,
1244554 characters, and 2050 entries.

Version 3.3.2, Mar 20 1996: A number of new entries pursuant on 3.3.2. This
version had 24442 lines, 190867 words, 1262468 characters, and 2061 entries.

Version 3.3.3, Mar 25 1996: Cleanup before TNHD III manuscript freeze. This
version had 24584 lines, 191932 words, 1269996 characters, and 2064 entries.

Version 4.0.0, Jul 25 1996: The actual TNHD III version after copy-edit. This
version had 24801 lines, 193697 words, 1281402 characters, and 2067 entries.

Version 4.1.0, 8 Apr 1999: The Jargon File rides again after three years. This
version had 25777 lines, 206825 words, 1359992 characters, and 2217 entries.

Version 4.1.1, 18 Apr 1999: Corrections for minor errors in 4.1.0, and some new
entries. This version had 25921 lines, 208483 words, 1371279 characters, and
2225 entries.

Version 4.1.2, 28 Apr 1999: Moving texi2html out of the production path. This
version had 26006 lines, 209479 words, 1377687 characters, and 2225 entries.

Version 4.1.3, 14 Jun 1999: Minor updates and markup fixes. This version had
26108 lines, 210480 words, 1384546 characters, and 2234 entries.

Version 4.1.4, 17 Jun 1999: Markup fixes for framed HTML. This version had
26117 lines, 210527 words, 1384902 characters, and 2234 entries.

Version 4.2.0, 31 Jan 2000: Markup fixes for framed HTML. This version had
26598 lines, 214639 words, 1412243 characters, and 2267 entries.

Version 5.0.0, 25 Jun 2011: The Jargon File comes alive yet again after an 8
year hiatus, maintained by Yash Tulsyan. Was forked from 4.2.0, revised in
light of criticisms against ESR-2003. This version has no word count as of yet.

Version 5.0.1, 5 Jan 2012: The Jargon File is being fixed and upsated, OSI
politics are replaced in favor of OSI-FSF agnostic politics.  This version has
no word count as of yer.

Version numbering: Version numbers should be read as major.minor.revision.
Major version 1 is reserved for the `old' (ITS) Jargon File, jargon-1. Major
version 2 encompasses revisions by ESR (Eric S. Raymond) with assistance from
GLS (Guy L. Steele, Jr.) leading up to and including the second paper edition.
Usually later versions will either completely supersede or incorporate earleir
versions, so there is generally no point in keeping old versions around. One
should also remember that due to the long hiatusen between major versions,
there have been many forks. We hope that if you find issue with our version,
you will send us suggestions to improve it. If, however, you challenge the
legitimacy of this edition, then by all means fork it.

Many thanks to ESR, who maintained ESR-2003, and I renew his statement of
thanks that follows:

Our thanks to the coauthors of Steele-1983 for oversight and assistance, and to
the hundreds of Usenetters (too many to name here) who contributed entries and
encouragement. More thanks go to several of the old-timers on the Usenet group
alt.folklore.computers, who contributed much useful commentary and many
corrections and valuable historical perspective: Joseph M. Newcomer
\url{<jn11+@andrew.cmu.edu>}, Bernie Cosell \url{<cosell@bbn.com>}, Earl
Boebert \url{<boebert@SCTC.com>}, and Joe Morris
\url{<jcmorris@mwunix.mitre.org>}.

We were fortunate enough to have the aid of some accomplished linguists, David
Stampe \url{<stampe@hawaii.edu>} and Charles Hoequist \url{<hoequist@bnr.ca>}
contributed valuable criticisms; Joe Keane \url{<jgk@osc.osc.com>} helped us
improve the pronunciation guides.

A few bits of this text quote previous works. We are indebted to Brian A.
LaMacchia \url{<bal@zurich.ai.mit.edu>} for obtaining permission for us to use
material from the \worktitle{TMRC Dictionary}; also, Don Libes
\url{<libes@cme.nist.gov>} contributed some appropriate material for his
excellent book \worktitle{Life With UNIX}. We thank Per Lindberg
\url{<per@front.se>}, author of the remarkable Swedish-language `zine'
\worktitle{Hackerbladet}, for bringing \worktitle{FOO!} comics to our attention
and smuggling one of the IBM hacker underground's own baby jargon files out to
us. Thanks also to Maarten Litmaath for generously allowing the inclusion of
the ASCII pronunciation guide he formerly maintained. And our gratitude to Mark
Weiser of XEROX PARC \url{<Marc_Weiser.PARC@xerox.com>} for securing us
permission to quote from PARC's own jargon lexicon and shipping us a copy.

It is a particular pleasure to acknowledge the major contributions of Mark
Brader \url{<msb@sq.com>} and Steve Summit \url{<scs@eskimo.com>} to the File
and Dictionary; they have read and reread many drafts, checked facts, caught
typos, submitted an amazing number of thoughtful comments, and done yeoman
service in catchign typos and minor usage bobbles. Their rare combination of
enthusiasm, persistence, wide-ranging technical knowledge, and precissionism in
matters of language has been of invaluable help. Indeed, the sustained volume
and quality of Mr. Brader's input over several years and several different
editions has only allowed him to escape co-editor credit by the slimmest of
margins.

Finally, George V. Reilly \url{<georger@microsoft.com>} helped with the TeX
arcana and painstakingly proofread some 2.7 and 2.8 versions, and Eric
Tiedemann \url{<est@thyrsus.com>} contributed sage advice throughout on
rhetoric, amphigory, and philosophunculism.

