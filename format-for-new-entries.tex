You can mail submissions for the Jargon File to \url{yashtulsyan@gmail.com}.

We welcome new jargon, and corrections to or amplifications of existing
entries. You can improve your submission's chances of being included by adding
background information on user population and years of currency. References to
actual usage via URLs and/or DejaNews pointers are particularly welcomed.

All contributions and suggestions about the Jargon File will be considered
donations to be placed in the public domain as part of this File, and may be
used in subsequent paper editions. Submissions may be edited for accuracy,
clarity and concision.

We are looked to expand the File's range of technical specialties covered.
There are doubtless rich veins of jargon yet untapped in the scientific
computing, graphics, and networking hacker communities; also in numerical
analysis, computer architectures and VLSI design, language design, and many
other related fields. Send us your jargon!

We are not interested in straight technical terms explained by textbooks or
technical dictionaries unless an entry illuminates `underground' meanings or
aspects not covered by official histories. We are also not interested in `joke'
entries -- there is a lot of humor in the file but it most flow naturally out
of the explanations of what hackers do and how they think. Furthermore, we will
emphatically reject all entries used to promote a political or religious agenda
unless it's justified by a large proportion of hackers (e.g. the Free Software
movement will get into the Jargon File, but libertarianism, which does not
account for a majority of hackers, would not). This is one of the reasons why
version 5.0.0 was forked from ESR-2003.

It is OK to submit items of jargon you have originated if they have spread to
the point of being used by people who are not personally acquainted with you.
We prefer items to be attested by independent submission from two different
sites.

Please send URLs for materials related to the entries, so we can enrich the
File's link structure.

The Jargon File will be regularly maintained and made available for browsing on
the World Wide Web, and will include a version number. Read it, pass it around,
contribute -- this is your monument!

