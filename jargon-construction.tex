There are some standard methods of jargonification that became established quite early (i.e., before 1970), spreading from such sources as
the Tech Model Railroad Club, the PDP-1 SPACEWAR hackers, and John McCarthy's original crew of LISPers. These include verb doubling,
soundalike slang, the `--P' convention, overgeneralization, spoken inarticulations, and anthropomorphization. Each is discussed below. We
also cover the standard comparitives for design quality.

Of these six, verb doubling, overgeneralization, anthropomorphization, and (especially) spoken inarticulatinos have become quite general;
but soundalike slang is still alrgely confined to MIT and other large universities, and the `--P' convention is found only where LISPers
flourish.

\begin{itemize}
	\item\citeentry{Verb Doubling}: Doubling a verb may change its semantics
	\item\citeentry{Soundalike Slang}: Punning jargon
	\item\citeentry{The `--P' convention}: A LISPy way to form questions
	\item\citeentry{Overgeneralization}: Standard abuses of grammar
	\item\citeentry{Spoken Inarticulations}: Sighing and $<$*sigh$>$ing
	\item\citeentry{Anthropomorphization}: Homunculi, daemons, and confused programs
	\item\citeentry{Comparatives}: Standard comparatives for design quality
\end{itemize}


\section*{Verb Doubling}\label{Verb-Doubling}
	A standard construction in English is to double a verb and use it as an exclamation, such as ``Bang, bang!'' or ``Quack, quack!''. Most
	of these are names for noises. Hackers also double verbs as a concise, sometimes sarcastic comment on what the implied subject does.
	Also, a doubled verb is often used to terminate a conversation, in the process remarking on the current state of affairs or what the
	speaker intends to do next. Typical examples involve \citeentry{win}, \citeentry{lose}, \citeentry{hack}, \citeentry{flame},
	\citeentry{barf}, \citeentry{chomp}:

	\begin{quote}
		``The disk heads just crashed.'' ``Lose, lose.''\\
		``Mostly he talked about his latest crock. Flame, flame.''\\
		``Boy, what a bagbiter! Chomp, chomp!''
	\end{quote}

	Some verb-doubled constructions have special meanings not immediately obvious from the verb. These have their own listings in the
	lexicon.

	The \citeentry{Usenet} culture has one tripling convention unrelated to this: the names of `joke' topic groups often have a tripled
	last element. The first and paradigmatic example was alt.swedish.chef.bork.bork.bork (a \worktitle{Muppet Show} reference); other
	infamous examples have included:

	\begin{quote}
		alt.french.captain.borg.borg.borg\\
		alt.wesley.crusher.die.die.die\\
		comp.unix.internals.system.calls.brk.brk.brk\\
		sci.physics.edward.teller.boom.boom.boom\\
		alt.sadistic.dentists.drill.drill.drill
	\end{quote}


\section*{Soundalike Slang}\label{Soundalike-Slang}

	Hackers will often make rhymes or puns in order to convert an ordinary word or phrase into something more interesting. It is considered
	particularly \citeentry{flavorful} if the phrase is bent so as to include some other jargon word; thus the computer hobbyist magazine
	\worktitle{Dr. Dobb's Journal} is almost always referred to among hackers as `Dr. Frob's Journal' or simply `Dr. Frob's'. Terms of
	this kind that have been in fairly wide use include names for newspapers:

	\begin{quote}
		Boston Herald $\Rightarrow$ Horrid (or Harried)\\
		Boston Globe $\Rightarrow$ Boston Glob\\
		Houston (or San Francisco) Chronicle $\Rightarrow$ the Crocknicle (or the Comical)\\
		New York Times $\Rightarrow$ New York Slime\\
		Wall Street Journal $\Rightarrow$ Wall Street Urinal
	\end{quote}

	However, terms like these are often made up on the spur of the moment, Standard examples include:

	\begin{quote}
		Data General $\Rightarrow$ Dirty Genitals\\
		IBM 360 $\Rightarrow$ IBM Three-Sickly\\
		Government Property -- Do Not Duplicate (on keys) $\Rightarrow$ Government Duplicity -- Do Not Propagate\\
		for historical reasons $\Rightarrow$ for hysterical raisins\\
		Margaret Jacks Hall (the CS building at Stanford) $\Rightarrow$ Marginal Hacks Hall\\
		Microsoft $\Rightarrow$ Microsloth\\
		Internet Explorer $\Rightarrow$ Internet Exploiter
	\end{quote}

	This is not really similar to the Cockney rhyming slang it has been compared to in the past, because Cockney substitutions are opaque
	whereas hacker punning jargon is intentionally transparent.


\section*{The `--P' convention}\label{The-P-convention}

	Turning a word into a question by appending the syllable `P'; from the LISP convention of appending the letter `P' to denote a
	predicate (a boolean-valued function). The question should expect a yes/no answer, though it needn't. (See \citeentry{T} and
	\citeentry{NIL}.)

	At dinnertime:

	\begin{quote}
		Q: ``Foodp?''

		A: ``Yeah, I'm pretty hungry.'' or ``T!''
	\end{quote}

	At any time:

	\begin{quote}
		Q: ``State-of-the-world-P?''

		A: (Straight) ``I'm about to go home.''

		A: (Humorous) ``Yes, the world has a state.''
	\end{quote}

	On the phone to Florida:

	\begin{quote}
		Q: ``State-p Florida?''

		A: ``Been reading JARGON.TXT again, eh?''
	\end{quote}

	[One of the best of these is a \citeentry{Gosperism}. Once, when we were at a chinese restaurant, Bill Gosper wanted to know whether
	someone would like to share with him a two-person-sized bowl of soul. His inquiry was: ``Split-p soup?'' -- GLS]


\section*{Overgeneralization}\label{Overgeneralization}

	A very conspicuous feature of jargon is the frequency with which techspeak items such as names of program tools, command language
	primitives, and even assembler opcodes are applied to contexts outside of computing wherever hackers find amusing analogies to them.
	Thus (to cite one of the best-known examples) Unix hackers often \citeentry{grep} for things rather than searching for them. Many of
	the lexicon entries are generalizations of exactly this kind.

	Hackers often enjoy overgeneralization on the grammatical level as well. Many hackers love to take various words and add the wrong
	endings to them to make nouns and verbs, often by extending a standard rule to nonuniform cases (or vice verse). For example, because

	\begin{quote}
		porous $\Rightarrow$ porosity\\
		generous $\Rightarrow$ generosity
	\end{quote}

	hackers happily generalize:

	\begin{quote}
		mysterious $\Rightarrow$ mysteriosity\\
		ferrous $\Rightarrow$ ferrosity\\
		obvious $\Rightarrow$ obviosity\\
		dubious $\Rightarrow$ dubiosity
	\end{quote}

	Another class of common construction uses the suffix `--itude' to abstract a quality from just about any adjective or noun. This usage
	arises especially in cases where mainstream English would perform the same abstraction through `--iness' or `--ingness'. Thus:

	\begin{quote}
		win $\Rightarrow$ winnitude (a common exclamation)\\
		loss $\Rightarrow$ lossitude\\
		cruft $\Rightarrow$ cruftitude\\
		lame $\Rightarrow$ lamitude
	\end{quote}

	Some hackers cheerfully reverse this transformation; they argue, for example, that the horizontal degree lines on a globe ought to be
	called `lats' -- after all, they're measuring latitude!

	Also, note that all nouns can be verbed. E.g.: ``All nouns can be verbed'', ``I'll mouse it up'',  ``Hang on while I clipboard it
	over'', ``I'm grepping the files''. English as a whole is already heading in this direction (towards pure-positional grammar like
	Chinese); hackers are simply a bit ahead of the curve.

	The suffix ``--full'' can also be applied in generalized and fanciful ways, as in ``As soon as you have more than one cachefull of
	data, the system starts thrashing,'' or ``As soon as I have more than one headfull of ideas, I start writign it all down.'' A common
	use is a ``screenfull'' meaning the amount of text that will fit in one screen, usually in text mode when you have no choice as to
	character size. Another common form is ``bufferfull''.

	However, hackers avoid the unimaginative verb-making techniques characteristic of marketroids, bean-counters, and the Pentagon: a
	hacker would never, for example, `productize', `prioritize', or `securitize' things. Hackers have a strong aversino to bureaucratic
	bafflegab and regard those who use it with contempt. The terms for `free software' are an example -- `free software' is the original,
	most hackish term, and those who use it are a significant minority. It calls to mind freedom, and most who use it also use copyleft
	licenses, such as the GNU General Public License. ESR coined the term `open-source' in order to appeal to businesses, and those who use
	it are a majority of hackers and hacker-friendly (or at least hacker-saturated) businesses. The GNU GPL is popular here as well, but
	copycenter licenses are gaining in popularity. Then businesses coined the bureaucratic buzzword `crowdsourcing', in reference to
	outsourcing, and this is rare amongst hackers, at least when talking of software.

	Similarly, all verbs can be nouned. This is only a slight overgeneralization in modern English; in hackish, however, it is good form to
	mark them in some standard nonstandard way. Thus:

	\begin{quote}
		win $\Rightarrow$ winnitude, winnage\\
		disgust $\Rightarrow$ disgustitude\\
		hack $\Rightarrow$ hackification
	\end{quote}

	Further, note the prevalence of certain kinds of nonstandard plural forms. Some of these go back quite a ways: the TMRC Dictionary
	includes an entry which implies that the plural of `mouse' is \citeentry{meeces}, and notes that the defined plural of `caboose' is
	`cabeese'. This latter has apparently been standard (or at least a standard joke) among railfans (railroad enthusiasts) for many years.

	On a similarly Anglo-Saxon note, almost anything ending in `x' may form plurals in `--xen' (see \citeentry{VAXen} and
	\citeentry{boxen} in the main text). Even words ending in phonetic /k/ alone are sometimes treated this way; e.g., `soxen' for a bunch
	of socks. Other funny plurals are `frobbotzim' for the plural of `frobbozz' (see \citeentry{frobnitz}) and `Unices' and `Twenices'
	(rather than `Unixes' and `Twenexes'; see \citeentry{Unix}, \citeentry{TWENEX} in main text). But note that `Unixen' and `Twenexen' are
	never used; it has been suggested that this is because `--ix' and `--ex' are Latin singular endings that attract a Latinate plural.
	Finally, it has been suggested to general approval that the plural of `mongoose' ought to be `polygoose'.

	The pattern here, as with other hackish grammatical quirks, is generalization of an inflectional rule that in English is either an
	import or a fossil (such as the Hebrew plural ending `--im', or the Anglo-Saxon plural suffix `--en') to cases where it isn't normally
	considered to apply.

	This is not `poor grammar' (which is itself a somewhat barmy idea), as hackers are generally quite well aware of what they are doing
	when they distort the language. It is grammatical creativity, a form of playfulness. It is done not to impress but to amuse, and never
	at the expense of clarity.


\section*{Spoken Inarticulations}\label{Spoken-Inarticulations}

	Words such as `mumble', `sigh', and `groan' are spoken in places where their referent might more naturally be used. It has been
	suggested that this usage derives from the impossibility of representing such noises on a comm link or in electronic mail, MUDs, and
	IRC channels (interestingly, the same sorts of constructions have been showing up with increasing frequency in comic strips). Another
	expression sometimes heard is ``Complain!'', meaning ``I have a complaint!''


\section*{Anthropomorphization}\label{Anthropomorphization}

	Semantically, one rich source of jargon construction is the hackish tendency to anthropomorphize hardware and software. This isn't done
	in a naive way; hackers don't personalize their stuff in the sense of feeling empathy with it, nor do they mystically believe that the
	things they work on every day are `alive'. What is common is to hear hardware or software talked about as though it has homunculi
	talking to each other inside it, with intentions and desires. Thus, one hears ``The protocol handler got confused'', or that programs
	``are trying'' to do things, or one may say of a routine that ``its goal in life is to X''. One even hears explanations like``\dots
	and its poor little brain couldn't understand X, and it died.'' Sometimes modelling things this way actually seems to make them easier
	to understand, perhaps because it's instinctively natural to think of anything with a really complex behavioral repertoire as `like a
	person' rather than `like a thing'.


\section*{Comparatives}\label{Comparatives}

	Finally, note that many words in hacker jargon have to be understood as members of sets of comparatives. This is especially true of the
	adjectives and nouns used to describe the beauty and functional quality of code. Here is an approximately correct spectrum:

	\begin{quote}
		monstrosity brain-damage screw bug lose misfeature\\
		crock kluge hack win feature elegance perfection
	\end{quote}

	The last is spoken of as a mythical absolute, approximated but never actually attained. Another similar scale is used for describing
	the reliability of software:

	\begin{quote}
		broken flaky dodgy fragile brittle\\
		solid robust bulletproof armor-plated
	\end{quote}

	Note, however, that `dodgy' is primarily Commonwealth Hackish (it is rare in the U.S.) and may change places with `flaky' for some
	speakers.

	Coinages for describing \citeentry{lossage} seem to call forth the very finest in hackish linguistic inventiveness; it has been truly
	said that hackers have even more words for equipment failure than Yiddish has for obnoxious people.

