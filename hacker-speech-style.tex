Hackish speech generally features extremely precise diction, careful word choice, a relatively large working vocabulary, and relatively
little use of contractions or street slang. Dry humor, irony, puns, and a mildly flippant attitude are highly valued -- but an underlying
seriousness and intelligence are essential. One should use just enough jargon to communicate precisely and identify oneself as a member of
the culture; overuse of jargon or a breathless, excessively gung-ho attitude is considered tacky and the mark of a loser.

This speech style is a variety of the precisionist English normally spoken by scientists, design engineers, and academics in technical
fields. In contrast with the methods of jargon construction, it is fairly constant throughout hackerdom.

It has been observed that many hackers are confused by negative questions -- or, at least, that the people to whom they are talking are
often confused by the sense of their answers. The problem is that they have done so much programming that distinguishes between

\begin{verbatim}
	if (going) ...
\end{verbatim}

and

\begin{verbatim}
	if (!(going)) ...
\end{verbatim}

that when they parse the question ``Aren't you going?'' it may seem to be asking the opposite question from ``Are you going?'', and so to
merit an answer in the opposite sense. This confuses English-speaking non-hackers because they were taught to answer as though the negative
part weren't there. In some other languages (including Russion, Chinese, and Japanese) the hackish interpretation is standard and the
problem wouldn't arise. Hackers often find themselves wishing for a word like French `si', German `doch', or Dutch `jawel' -- a word with
which one could unambiguously answer `yes' to a negative question. (See also \citeentry{mu})

For similar reasons, English-speaking hackers almost never use double negatives, even if they live in a region where colloquial usage
allows them. The thought of uttering something that logically ought to be an affirmative knowing it will be misparsed as a negative tends
to disturb them.

In a related vein, hackers sometimes make a game of answering questions containing logical connectives with a strictly literal rather than
colloquial interpretation. A non-hacker who is indelicate enough to ask a question like ``So, are you working on finding that bug now or
leaving it until later?'' is likely to get the perfectly correct answer ``Yes!'' (that is, ``Yes, I'm doing it either now or later, and
you didn't ask which!'').

